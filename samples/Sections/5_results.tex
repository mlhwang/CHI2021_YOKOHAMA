\vspace{-5pt}
\section{Results}

In the results, we focus around three key subjects: (1) users' general experiences with the Monster Munch app; (2) the importance of gamification mechanisms in shaping users' experiences with the app; and (3) the role of gamification mechanisms in facilitating nutrition recall, reflection, and learning. (\textit{All tables for all analyses are provided in the supplementary material for full description and inspection}.)

\vspace{-5pt}
\subsection{Participant Background}
Sixty eight participants completed the pilot study. Participants' ages ranged from 18 to 65 with a median age of 25-34 years old. Participants were evenly split across gender (44\% female) and education (50\% above high school/GED). Most participants reported not working professionally in nutrition (75\%) and having no prior experiences with using health applications (60\%). In choosing which monster to select, participants preferences were relatively evenly distributed across each of the four monster options: Monster 1 with `Type 2 Diabetes' (27.9\%), Monster 2 for `Weightloss' (27.9\%), Monster 3 for `Running a Marathon' (27.9\%), and Monster 4 with `Irritable Bowel Syndrome' (16.2\%). 

%\textcolor{red}{We should also include the corresponding nutritional goals. Or instead of the conditions?}

\vspace{-5pt}
\subsection{General Experiences with Monster Munch}

Overall, users liked experiencing the app and found it a useful tool for reflecting on nutrition. One participant reflected this sentiment by commenting: \textit{``This was fun and I think someone can learn a lot about nutrition through trial and error via this [app] instead of using their own body as an experiment''} (ID-072). Additionally, we discovered that users self-reported confidence in assessing macronutrients from meal photographs statistically significantly improved after a single interaction with the app (Paired T-Test, pretest - post-test; \textit{t}= - 3.48, \textit{p}=.001, Cohen's \textit{d} ES= 0.225).

Users spent an average of 217 seconds (approx. 3.5 min) in the app (\textit{SD}=129sec). Regarding time spent with the app, one user wanted to experience the app for an extended period and expressed there may be potential with a better narrative: \textit{``My investment might have been higher if it was longer and there was more [continuity]...Overall great [app] with great potential. Perhaps if the narrative and continuity is further developed''} (ID-243). 

%\textcolor{blue}{Wasn't this a DCP user not a PP user?}

\vspace{-5pt}
\subsection{Gamification and the Monster Munch Experience}
Next, we sought to understand how the gamification mechanisms of the app contributed to users' experience and enjoyment of the app. 
\vspace{-5pt}
\subsubsection {Avatar and User Experience}
We were interested in how users interacted with avatars, and how that in turn shaped their experiences with Monster Munch. Users reported their primary reasons for selecting the avatar to be: because they or someone they knew were working on the same goal, because they wanted to learn more about the health goal, and because they liked the appearance of the monster.

We assessed the degree users identified with their avatar through a series of questions calculated into a composite ``PAID score'' that measured users' affinity with their avatar (refer to section 4.3.1). 
User's PAID score did not vary significantly across any demographic categories other than education; with regards to education users who reported having a high school education or less identified significantly more with their avatars (Independent T-Test; \textit{p}=0.003, Cohen's \textit{d} ES=0.77). 
Then, we examined  how users' affinity towards their avatars related to their enjoyment of the game. We found stronger identification with the avatar significantly predicted app enjoyment (B=0.11, \textit{p}<0.001), app recommendation (B=0.92, \textit{p}< 0.001), and app rating (B=0.22, \textit{p}<0.001).
Lower educational background was associated with higher app enjoyment (\textit{p}=0.021, Cohen's \textit{d} ES=0.59), but was not associated with the rating and recommendation of the app.
Further, avatar identification was not significantly correlated with users' overall meal choice performance within the app, even after controlling for prior nutritional knowledge and educational background (\textit{p}=0.079). 

\vspace{-5pt}
\subsubsection {Community Board and User Experience}
We were also interested in how the second gamification mechanism embedded into the app design, the crowdsourced CB, might shape user's experience.
Broadly, user experience with the CB did not vary significantly across any demographic categories. However, we observed that user's self-reported influence by the CB's information significantly predicted their enjoyment of the app (B=0.267, \textit{p}=0.001), how they rated the app (B=0.469, \textit{p}=0.011), and willingness to recommend the app to others (B=0.281, \textit{p}<0.001).

\vspace{-5pt}
\subsection{Avatar Identification and  Nutrition Recall}

Although nutritional literacy or learning was a secondary, exploratory focus on this feasibility-oriented pilot study, the inclusion of a baseline and post-task nutritional assessment allowed us to investigate preliminary measurements to see if Monster Munch might relate to improved recollection of nutritional information and indicate that the app has the potential to become a future learning tool. We compared pre-task nutrition macronutrient assessment scores to performance on the same questions post-task and found significant improvement in the post-task evaluation after completing the Monster Munch app (Paired T-Test; \textit{t}=-2.392, \textit{p}=.020, Cohen's \textit{d} {ES=0.28}).

%We then looked to see how the gamification components might relate to performance on the post-test. 
We found that lower player-avatar identification marginally predicted greater recall of nutrition information, even after controlling for baseline nutrition knowledge, educational background, and Monster Munch task performance (\textit{B}= -0.009, \textit{p}= 0.051). Additionally, while both pre- and post-score nutritional assessment  predicted recall, there was no significant interaction between individuals pre-score assessment and avatar identification (\textit{p}=0.32). 
%Through these findings avatar identification emerges as an important possible predictor of nutritional knowledge recall. 
Looking at other gamification elements, users who indicated greater use of the community board performed significantly worse on cursory measures for recall after controlling for prior nutritional knowledge (B=-0.036 , \textit{p}=0.006).

%Unsurprisingly, among the 68 participants only a handful left such feedback for the research team, as it was an optional question at the very end of the entirety of the study. 


% One user from each phase mentioned wanting to have the recipes of the meal photographs that were shown to them. 