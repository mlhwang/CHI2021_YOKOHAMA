\section{Introduction}
Unhealthy eating contributes to the obesity epidemic in the United States, which affects 12 million American adolescents and 38\% of adults \cite{trustforamericashealth,cdc2015,cdc2020}. 
%The estimated annual medical cost of obesity exceeded \$147 billion dollars in 2008 [PLACE UPDATED REF]~\cite{ }. 
While there are many factors that contribute to unhealthy eating, past work highlights low nutritional literacy as a key factor, in particular limited ability to accurately estimate nutritional composition (i.e., macronutrient content) of meals ~\cite{kindig2004health}. 
While estimating the nutritional composition of meals is crucial for health management and preventing undesired health consequences (especially for individuals with chronic conditions), previous studies show doing so is a considerable challenge. For example, individuals from low-literacy backgrounds often have difficulty interpreting Nutrition Fact labels, leading to miscalculation of the amount of consumed nutrients \cite{chaudry2013formative,huizinga2009literacy,rothman2006patient}. Even professional dietitians~\cite{chandon2007obesity} and healthy eaters often underestimate calories in meals~\cite{chandon2007biasing}. Complex meals such as salads and dressings with many components add another layer of difficulty with nutritional assessment~\cite{noronha2011platemate}.


Due to the difficulty in nutritional estimation, as well as the huge user burden with many self-tracking and food journaling apps, people develop short-lived commitments to these tools and often are fatigued leading to complete abandonment of the app and self-tracking habits~\cite{choe2014understanding,Cordeiro:2015:RMF:2702123.2702154,cordeiro2015barriers,epstein2016crumbs,mattila2008mobile}. A variety of \textit{interactive, gamified solutions} have emerged to help individuals engage with nutrition information easily, promote meal tracking and behavior change, and improve learning. Noting the high burden of tracking ingredients and meals many users report~\cite{desai2019personal}, some solutions automate the process of nutrition logging 
through bar code scanning and ingredient selection from existing databases ~\cite{beijbom2015menu,bomfim2018pirate,bomfim2020food,siek2009evaluation}, lightweight social interactive challenges~\cite{Cordeiro:2015:RMF:2702123.2702154,epstein2016crumbs},
computational image analysis~\cite{anthimopoulos2015computer,kong2012dietcam,rhyner2016carbohydrate,zhang2015snap,zhu2010use}, crowdsourcing~\cite{mamykina2011examining,noronha2011platemate}, and immersive avatar-based gaming or tamagotchi-style  nurturing~\cite{ahn2017immersive,byrne2012caring,hwang2017monster,lin2006fish}.


While avatars are a commonly utilized gamification mechanism, there is a dearth of literature that has explored the relationship between player-avatar-identification and learning outcomes. A plethora of research has focused around how higher engagement, presence, and enjoyment of and in a game can be facilitated with more options to customize avatars~\cite{ahn2017immersive,bailey2009avatar,birk2016fostering,li2013player,trepte2010avatar,turkay2014effects,turkay2015effects}. Naturally, connecting greater presence and enjoyment in a game to learning is typically the next step in (educational) games research~\cite{de2019algebright,huizenga2009mobile,lin2017character,lin2019evaluating,ng2013examining,vogel2006computer}. However, there are seldom examples where the avatar customization itself and therefore the avatar's appearance embodies a learning concept. In our app, Monster Munch, the pet monster avatars' appearance changes from unhealthy to healthy depending on what meals are fed to them by the user and therefore delivers information about the meal itself and its health consequences. In other words, the appearance of the avatars is intended to do more than incentivize engagement but be an important part of the learning. In addition, crowdsourcing, a typical feature commonly categorized as social, is incorporated into our app to deliver nutritional information through what is called a community board (see Figure~\ref{fig:screenshots}, showing what the crowd fed their monster avatars to achieve the same nutritional goal the present user is trying to achieve for her/his pet avatar.

In this feasibility-focused, pilot study, we tested gamification mechanisms that have received relatively positive user feedback respectively, but have not been mixed together due to challenges in current approaches to nutritional assessment, food journaling, and access to ``in-the-wild'' crowdsourced meal photographs.

Specifically, we developed a mobile application, Monster Munch, where users help a self-selected monster avatar with a particular health goal (e.g., aiming to run a marathon) achieve that health goal by selecting a crowdsourced ``in-the-wild'' meal photograph that fits their monster's needs. When selecting the meal to feed their monster, users viewed meal photos and reviewed the meal selection reasoning of other members of the crowd through a community board (See Figure~\ref{fig:screenshots}). This work examines gamification and social mechanisms in the context of nutrition, closes gaps around engaging people with nutrition information in a simple, fun, crowdsourced, and personalized fashion. Our work examines gamification and social mechanisms in the context of nutrition, taking on the challenge of delivering nutrition literacy in an engaging, fun, crowdsourced, and personalized fashion.

In our pilot study (N=68), we observed that users' self-reported confidence in assessing macronutrients based on meal photographs had increased significantly after using the app. Strong player-avatar-identification (PAID) and increased utilization of the crowdsourced community board (where the crowd had voted and reasoned with what they considered the ``best meal'' of the available options was) were related to users' enjoyment, rating, and recommendation of the app to others. Interestingly, lower PAID predicted greater recall of nutrition information, independent of prior nutrition knowledge. 
%Findings suggest PAID may be an important mechanism in learning and highlights how fun, lightweight tools can prompt reflection and recall.

%\subsection{Contributions}
In summary, the key contributions of this paper are as follows:

(1) This work integrates previously disconnected gamification and social mechanisms (tamagotchi-style mechanism and crowdsourced community board) to facilitate engagement with and deliver nutritional information in a new context;

(2) This paper shows that a lightweight app such as Monster Munch has the potential to engage users and help them improve nutritional literacy and can be established and maintained by the crowd and for the crowd; 

(3) This study emphasizes the role that player-avatar-identification may have in shaping user's engagement and enjoyment of the app and possible effects on recollection and learning outcomes.




