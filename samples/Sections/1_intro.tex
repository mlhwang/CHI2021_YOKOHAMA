\section{Introduction}
Unhealthy eating is one of the main causes of the obesity epidemic in the United States, which affects 12 million American adolescents and 38\% of adults \cite{trustforamericashealth,cdc2015,cdc2020}. Past work has elucidated the impact of low nutritional literacy as a key contributor to unhealthy eating and has identified the important role that accurately estimating the nutritional composition (i.e., macronutrient content) of meals plays~\cite{kindig2004health}. Despite the crucial role estimating the nutritional composition of meals for health management plays in preventing undesired health consequences (especially for individuals with chronic conditions), previous studies show the considerable challenges to accurate nutritional estimation. For example, individuals from low-literacy backgrounds often have difficulty interpreting Nutrition Fact labels, leading to miscalculation of the amount of consumed nutrients \cite{chaudry2013formative,huizinga2009literacy,rothman2006patient}. Even professional dietitians~\cite{chandon2007obesity} and healthy eaters often underestimate calories in meals~\cite{chandon2007biasing}. Furthermore, complex meals such as salads and dressings with many components add another layer of difficulty with nutritional assessment~\cite{noronha2011platemate}.


Due to the difficulty of nutritional estimation, numerous self-tracking and food journaling apps were developed. These apps were meant to provide users with practice in estimating the nutritional content of their meals with the aim of improving skills over time. However, these tools present a huge user burden and many users develop short-lived commitments to these tools that ultimately lead to little knowledge increases and complete abandonment of the app~\cite{choe2014understanding,Cordeiro:2015:RMF:2702123.2702154,cordeiro2015barriers,epstein2016crumbs,mattila2008mobile}. As a result, a variety of \textit{interactive, gamified solutions} have emerged to help individuals engage with nutrition information more easily, promote meal tracking and behavior change, and improve learning. 

%Noting the high burden of tracking ingredients and meals reported by users~\cite{desai2019personal}, some solutions automate the process of nutrition logging 
% through bar code scanning and ingredient selection from existing databases ~\cite{beijbom2015menu,bomfim2018pirate,bomfim2020food,siek2009evaluation}, lightweight social interactive challenges~\cite{Cordeiro:2015:RMF:2702123.2702154,epstein2016crumbs},
% computational image analysis~\cite{anthimopoulos2015computer,kong2012dietcam,rhyner2016carbohydrate,zhang2015snap,zhu2010use}, crowdsourcing~\cite{mamykina2011examining,noronha2011platemate}, and immersive avatar-based gaming or tamagotchi-style  nurturing~\cite{ahn2017immersive,byrne2012caring,hwang2017monster,lin2006fish}.

Myriad gamified approaches from HCI and education have been applied to improve engagement with different subject matters including health and nutrition. A plethora of research has focused on how higher engagement, presence, and enjoyment of and in a game can be facilitated with increased customization options for avatars~\cite{ahn2017immersive,bailey2009avatar,birk2016fostering,li2013player,trepte2010avatar,turkay2014effects,turkay2015effects}. The literature has shown clear connections in (educational) games between greater presence and enjoyment in a game and learning ~\cite{de2019algebright,huizenga2009mobile,lin2017character,lin2019evaluating,ng2013examining,vogel2006computer}. However, there are few examples where the customization is focused on the avatar and even fewer where the avatar's appearance embodies a learning concept. 
In our app, Monster Munch, the pet monster avatars' appearance changes from ``unhealthy'' to ``healthy'' depending on what meals are fed to them by the user, delivering information about the meal and its potential health consequences clearly and tangibly to the user. Thus, the appearance of the avatars does more than incentivize engagement, but is also an important part of the user's learning. In addition, crowdsourcing, a typical feature commonly categorized as social, is incorporated into our app to deliver nutritional information through a community board (see Figure~\ref{fig:screenshots}). It is hypothesized that showing what the crowd fed their monster avatars to achieve the same nutritional goal the present user is trying to achieve for their pet avatar will impact users' meal choices.

In this feasibility-focused, pilot study, we tested gamification and social mechanisms that have received relatively positive user feedback respectively, but have not been combined due to challenges in current approaches to nutritional assessment, food journaling, and access to ``in-the-wild'' crowdsourced meal photographs.

Specifically, we developed a mobile application, Monster Munch, where users help a self-selected monster avatar with a particular health goal (e.g., aiming to run a marathon) by selecting a crowdsourced ``in-the-wild'' meal photograph that fits their monster's needs. When selecting the meal to feed their monster, users viewed meal photos and reviewed the meal selection reasoning of other members of the crowd through a community board (See Figure~\ref{fig:screenshots}). This work examines gamification and social mechanisms in the context of nutrition through the novel application of approaches that are simple, fun, crowdsourced, and personalized. 

% Our work addresses

% gamification and social mechanisms in the context of nutrition, taking on the challenge of delivering nutrition literacy in an engaging, fun, crowdsourced, and personalized fashion.

In our pilot study (N=68), we observed that users' self-reported confidence in assessing macronutrients based on meal photographs had increased significantly after using the app. Strong player-avatar-identification (PAID) and increased utilization of the crowdsourced community board (where the crowd voted and reasoned with what they considered the ``best meal'' of the available options) were related to users' enjoyment, rating, and recommendation of the app to others. Learning was marginally associated with low identification with avatars, showing that further investigation is essential for measuring learning outcomes as a by-product of exploring gamification and social mechanisms.

In summary, the key contributions of this paper are as follows: (1) This work integrates 
both gamification and social mechanisms (tamagotchi-style avatar nurturing and crowdsourced community board) to facilitate delivering nutritional information in a new context; (2) Our findings emphasize the role that player-avatar-identification has in shaping users' engagement and enjoyment of a lightweight nutritional app such as Monster Munch; and (3) The work shows that this novel integration of gamification and social mechanisms may have implications for nutritional literacy and recollection.




% a crowdsourced platform established and maintained by the crowd


% and shows that a lightweight app such as Monster Munch has the potential to engage 
% %users and help them 
% users in nutritional literacy while established and maintained by the crowd and for the crowd.
% %; (3) This study emphasizes the role that player-avatar-identification may have in shaping users' engagement and enjoyment of the app and possible effects on recollection and learning outcomes.


% We found that our users enjoyed using the app and their self-reported confidence for assessing macronutrient content in meal photographs increased post-app usage.
% A key insight was that specifically, identification with the pet monster avatar, as well as engagement with the community board, was highly predictive of enjoyment, rating of the app, and recommending the app to others. 
% Learning was only moderately shown with those who did not identify with their avatars highly, showing that further investigation is due for measuring learning outcomes as a by-product of exploring gamification mechanisms.  