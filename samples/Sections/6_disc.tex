\section{Discussion}
SOME RANDOM TEXT INCLUDED HERE.

\subsection{Learning}

\subsection{Player Avatar Identificaiton}
Meaning of high player identification scores but low learning scores. What can we make of that?

\subsection{Crowdsourced Intelligence}

Community board from Phase 1 but do not forget now we have an even more massive community board from Phase 2. 

\subsection{Limitations}

Limitations of this research study in this particular scope is discussed in the following sections. 

\subsubsection{Limitations with the Gamification Features}

For the "unlock accessory" feature, players were not made aware accessories were available to ``upgrade'' one's monster avatar if four rounds and five rounds worth of meals were correctly fed to the avatar. 

There was also no final winning or losing status mentioned to the players. We probably should have made it clear that there were unlocking accessories avail and once you reach that level you have actually ``won'' the game (though we are not calling this officially a game). 



\subsubsection{The Quality of the Users' Responses -- threats to validity?}
How much can we trust the Turkers that they indeed only had up to a high school degree when we were doing purposeful sampling. 

How much can we guarantee that the post-test responses were not influenced by fatigue. (Maybe this is a good place to quickly mention the average duration people took to complete the entire study? -- first mention this in the Evaluation section when maybe talking about Participants)


\subsubsection{Determining the Gold Standard for ``in-the-wild'' Meal Photographs}
First of all, we involved quite a number of people to achieve this. But also the meal photographs are from our previous studies~\cite{} -- keep it anonymized-- and that means most meals are from a particular geographical population (Washington Heights). In creating a true ``crowdsourced intelligence'' driven community board we would have to do a much more comprehensive collection of meal photographs. Maybe using soical media sites like IG and TikTok might help?


Taken from \cite{cuthbert2019effects}:
``Secondly, this study had a small number of participants
which made the statistics very vulnerable to type II errors.
We tried to deal with this limitation by reporting effect sizes,
descriptive statistics, and all the p values, avoiding the typical
practice of sharing only statistically significant results.''

\subsection{Future Work}

How do we get to a truly crowdsourced community board that really represents all people and all types of food? Is that even possible? Could there be one place, one repository for this, or should there not be just one repository? How can we do this more at scale?

