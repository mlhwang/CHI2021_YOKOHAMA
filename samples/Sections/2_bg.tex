\vspace{-5pt}
\section{Background and Relevant Work}
There is considerable literature that illustrates the difficulty in correctly estimating the nutritional composition of meals\cite{berkman2011low,chandon2007obesity,chaudhry2016evaluation,chaudry2013formative,stanton2006nutrition,kim2014energy,lansky1982estimates,schwartz2006ability}. Frequently reported barriers of low nutritional literacy and poor numeracy skills make estimating the nutritional composition of meals difficult. Furthermore, macronutrients are typically taught in a way that focuses on each macronutrient individually. However, the application of these skills most frequently occurs in the context of complete meals, which are rarely presented by each individual macronutrient, creating misalignment between teaching and use in the real world. This has inspired active research in several fields to create nutritional education tools that teach users these skills based in longstanding educational and pedagogical traditions of observational learning.

\vspace{-5pt}
\subsection{Observational Learning Mechanisms for Improvement in Nutritional Literacy}
%It is well established that 
Humans can learn vicariously by observing other people's actions and their subsequent consequences. This concept of observational learning (OL) is supported through the educational
`learning-by-example’ paradigm~\cite{anderson1997role,atkinson2000learning,brown1988preschool} and Social Cognitive Theory~\cite{bandura1989human,bandura1998health}. OL has been utilized in various research fields, specifically for social computing platforms and nutrition. \cite{mamykina2016learning} examined accuracy gain in nutritional assessments of different meals with paid crowd workers by giving users both expert and peer-generated feedback. 
%Expert generated feedback helped in the crowd workers' accuracy gain of nutritional evaluation of meals. However, 
Learning and accuracy gain were %also 
observed when crowd workers could compare their own solutions to those provided by others. These findings support the `learning-by-example’ paradigm and how social computing crowdsourced platforms have potential to become an effective teaching tool. 

\cite{burgermaster2017role} examined gains in nutritional learning with unpaid crowd workers when comparing macronutrient content in pairs of meal photographs. This study showed that viewing peer responses or receiving correctness feedback alone did not result in any learning gains. However, when explanations were provided with correctness feedback, there were significant learning gains regardless of whether the feedback was provided by experts or peers. Both studies point to the promise of delivering nutrition content and education through social and casual learning environments. 

 
\vspace{-5pt}
\subsection{Computational and Crowdsourcing Approaches to Nutritional Estimation}

There are numerous examples of  interventions that combine crowdsourcing and recent advances in computing and automation to help alleviate the nutritional assessment burden on users. 
Early applications of computing and automation in this domain focused on decreasing the burden associated with capturing or tracking foods. Siek et al.~\cite{siek2006we} created an application that relied on selecting meals from nutritional databases or by scanning bar codes in pre-packaged foods. 

Some computational image analysis techniques include using machine learning algorithms to recognize components of meals based on training datasets of meal images labeled by experts or crowd workers~\cite{anthimopoulos2015computer,pouladzadeh2016food,rhyner2016carbohydrate,Thomaz:2013:FIE:2526667.2526672}. 
In later studies crowdsourcing and computing were leveraged together to create more comprehensive solutions. A study by Noronha et al.~\cite{noronha2011platemate}, authors specifically focused on developing crowdsourcing workflows that help crowd workers arrive at accurate nutritional composition of submitted meals. However, they found that this system struggled to produce good results on liquids like beverages and salad dressings. One study by \cite{merler2016snap} specifically aimed to create a large data set of ``in-the-wild'' food photos from crowdsourced mechanisms and created one of the largest food databases of real photos from real people. The limitation of this study was that even though these were real people's meal photos and a sufficiently large database, they were all single food items and food segmentation and portion estimation components were lacking to become useful for dietary assistance and nutritional assessment and evaluation. Together these studies identified clear useful applications of these approaches in addressing the challenges associated with nutritional estimation of meals. However, these approaches may benefit from adding gamification and gameful components to their workflow, as gameful design has shown to motivate the crowd to engage more intrinsically. Our research took these studies and added the gameful element to garner the community's interest in nutrition. 



\vspace{-5pt}
\subsection{Gameful Design Approaches for Improvement in Nutritional Literacy}
A number of studies have incorporated game elements in the field of health and specifically nutrition, as games and  gamification have shown to encourage players to engage in learning activities and material outside traditional learning environments \cite{alessi1991computer, blumberg2014serious,chen2014healthifying,huizenga2009mobile,jui2011game,mueller2011designing,papastergiou2009exploring,richards2014beyond}. In addition, gameful design has shown to improve health attitudes and behaviors in a number of studies including nutrition~\cite{deterding2011game,grimes2010let, johnson2016gamification,kyfonidis2019making,peng2009design,orji2013lunchtime,orji2017improving}.

Priate Bri's Grocery Adventure incorporated a situated, gameful approach to grocery shopping that sought to improve food literacy through a series of grocery shopping trips~\cite{bomfim2018pirate,bomfim2020food}. The study showed that users who utilized the app were successful in increasing their nutrition knowledge, shopping for healthier food items, and reducing their impulse buys. 

In Escape from Diab~\cite{thompson2008serious}, the main character trains others to increase their physical strength by healthy eating and exercising. Lunch Crunch~\cite{lunchcrunch} makes players place healthy items on lunch trays but trash unhealthy items to educate the players on (un)healthy options. In OrderUP!~\cite{grimes2010let}, one plays a server at a restaurant where she/he recommends her/his customers the healthy options to keep the job. In LunchTime~\cite{orji2013lunchtime} players can choose a health goal (i.e., manage weight, manage diabetes, manage blood pressure, build muscle, and general well-being) and choose meals from restaurants according to their health goals. More points are awarded if the healthiness of the meals align with the health goal. Both OrderUP! and LunchTime have shown to increase the players' nutrition knowledge and general feelings of self-efficacy related to nutrition behaviors. 

Monster Appetite~\cite{hwang2017monster} is a game that incorporates framed messages encouraging players to 1) make unhealthy choices to feed their avatar or 2) defend healthy choices and focus on the benefits of such choices. A user study comparing the two framed messages showed that when ``healthy'' messages were \textit{combined with negative visuals of the avatars}, participants were more likely to exhibit healthier snack choices in their simulated snack shop exercise. This study showed that the appearance of the avatar had an important role in which message resulted in positive outcomes.



\vspace{-5pt}
\subsubsection{Player Avatar Identification}
Customization can increase user's enjoyment~\cite{birk2016fostering,marathe2011drives,trepte2010avatar,turkay2015effects}, engagement and presence~\cite{ng2013examining}, motivation to play~\cite{turkay2015effects}, and facilitate self-expression in digital environments such as games~\cite{adinolf2011controlling,bailey2009avatar}. Unlike functional customizations (such as changing the difficulty levels of an enemy AI), cosmetic customizations do not affect the player's gameplay, but can improve one's self-expression and enjoyment via affecting the appearance of avatars and game objects, and, therefore game environments~\cite{cuthbert2019effects}. Previous examples of successful cosmetic customizations are hair colors and gender of avatar selection by the user~\cite{turkay2014effects}. 

Scholars have posited a human-animal bond in which pet owners feel their pet's psychological and physical health is related to their own well-being~\cite{chesney2007illusion,cohen2001defining,hosey2014human,lin2017exploring,zichermann2011gamification}. Several studies have shown that keeping oneself motivated to stay healthy in order to take care of a virtual pet have succeeded~\cite{ahn2015using,byrne2012caring,lin2006fish,pollak2010s}, which supports the special human-animal bond theory. To promote user engagement with Monster Munch, we incorporated pet avatars and both intrinsic and extrinsic motivators~\cite{chen2016scaffolding,habgood2011motivating} in our study. In our proposed app environment, a user can ``level up'' (progress to the next status) with a high enough meal score, to make their pet avatar happy and healthy. Users can also receive rewards such as hats, toys, and other virtual items (see Figure~\ref{fig:monster-stages}) to further encourage personalization and avatar identification.


Despite the growing application of avatars in tools for learning, there remains a gap in connecting the avatar identification mechanism to specific learning outcomes, especially in the context of nutrition~\cite{chang2019stereotype,chen2019effects,de2019algebright,lin2019evaluating}. In a simulation game based on the 2010 Haitian hurricane~\cite{bachen2016presence}, researchers found that female users who felt sympathy after playing this showed an increase in interest in learning more about the game topic offline. Similarly, numerous studies examine avatars and ``learning'' by examining more proximal outcomes such as self-efficacy, intention to learn, increased interest in the topic and context of the game, etc. However, none of the literature has investigated if the use of pet avatars and altering their appearance based on responding correctly will impact learning. It is hoped, serving the mechanism to create a nutritional teaching tool in the future. 



There are a plethora of studies involving players taking on a role and caring for an avatar as well as those that focus on increasing food and nutritional literacy, but studies focusing on player-avatar-identification/connection and how that may relate to nutritional engagement are limited. Our study is among the first, to the best of our knowledge, to look at the relationship of the tamagotchi-style mechanism of taking care of one's avatar and exploring whether that connection can lead to better engagement and learning within the nutritional context. 






