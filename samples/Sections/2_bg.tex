\vspace{-5pt}
\section{Background and Relevant Work}
The considerable difficulty of correctly estimating nutrition in meals is not limited to the task itself. It is also because resources that can help educate people are usually experts in the field who are costly and inaccessible to many. This has inspired active research in several fields to automate the process or leverage the crowd's intelligence. 

 
\vspace{-5pt}
\subsection{Computational and Crowdsourcing Approaches to Nutritional Estimation}

There exist rich literature describing challenges associated with nutritional estimation of meals \cite{berkman2011low,chandon2007obesity,chaudhry2016evaluation,chaudry2013formative,stanton2006nutrition,kim2014energy,lansky1982estimates,schwartz2006ability}. These challenges present a need for different interventions that may leverage computing, crowdsourcing, and automating that may help alleviate the burden of users when it comes to nutritional assessment and health-tracking. Some solutions rely on selecting meals from nutritional databases or by scanning bar codes in pre-packaged foods~\cite{siek2006we}. Some computational image analysis techniques include using machine learning algorithms to recognize components of meals based on training datasets of meal images labeled by experts or crowd workers~\cite{anthimopoulos2015computer,pouladzadeh2016food,rhyner2016carbohydrate,Thomaz:2013:FIE:2526667.2526672}. 
Others specifically focused on developing crowdsourcing workflows that help crowd workers arrive at accurate nutritional composition of submitted meals~\cite{noronha2011platemate}. 
 
However, this system struggled to produce good results on liquids like beverages and salad dressings. One study by \cite{merler2016snap} specifically aimed to create a large data set of ``in-the-wild'' food photos from crowdsourced mechanisms and created one of the largest food databases of real photos from real people. \textcolor{red}{TRANSITION SENTENCE}


\vspace{-5pt}
\subsection{Observational Learning Mechanisms for Improvement in Nutritional Literacy}
%It is well established that 
Humans can learn vicariously by observing other people's actions/decisions and their subsequent consequences. This concept of observational learning (OL) is supported through the educational
`learning-by-example’ paradigm~\cite{anderson1997role,atkinson2000learning,brown1988preschool} and Bandura's Social Cognitive Theory~\cite{bandura1989human,bandura1998health}. OL has been utilized in various research fields, specifically for social computing platforms and nutrition. \cite{mamykina2016learning} examined accuracy gains in nutritional assessment of different meals with paid crowd workers when they received expert feedback compared to peer-generated feedback. Expert generated feedback helped in the crowd workers' accuracy gain of nutritional evaluation of meals. However, learning and accuracy gain were also observed when crowd workers could compare their own solutions to those provided by others. These findings support the `learning-by-example’ paradigm and how social computing crowdsourced platforms have potential to become an effective teaching tool. 

\cite{burgermaster2017role} examined gains in nutritional learning with unpaid crowd workers when comparing macronutrient content in pairs of photographed meals. The study showed that viewing peer responses or receiving correctness feedback alone did not show any learning gains. However, when explanations were provided with correctness feedback, there were significant learning gains regardless of whether the feedback was provided by experts or peers. Both studies show promise in delivering nutrition content and education through social computing-based casual learning environments. However, these approaches may benefit from adding gamification and gameful components to their workflow, as gameful design has shown to motivate the crowd to engage more intrinsically. Our research took these studies and added the gameful element to garner the community's interest in nutrition. 

\vspace{-5pt}
\subsection{Gameful Design Approaches for Improvement in Nutritional Literacy}
A number of studies have incorporated game elements in the field of health and specifically nutrition, as games and  gamification in context have shown to encourage players to engage in learning activities and material outside traditional learning environments \cite{alessi1991computer, blumberg2014serious,chen2014healthifying,huizenga2009mobile,jui2011game,mueller2011designing,papastergiou2009exploring,richards2014beyond}. In addition, gameful design has shown to improve health attitudes and behaviors in a number of studies including nutrition~\cite{deterding2011game,grimes2010let, johnson2016gamification,kyfonidis2019making,peng2009design,orji2013lunchtime,orji2017improving}.

Priate Bri's Grocery Adventure incorporated a situated, gameful approach to grocery shopping that sought to improve food literacy through a series of grocery shopping trips~\cite{bomfim2018pirate,bomfim2020food}. The study showed that users who utilized the app were successful in increasing their nutrition knowledge, shopping for healthier food items, and reducing their impulse buys. 

In Escape from Diab~\cite{thompson2008serious}, the main character trains others to increase their physical strength by healthy eating and exercising. Lunch Crunch~\cite{lunchcrunch} makes players place healthy items on lunch trays but trash unhealthy items to educate the players on (un)healthy options. In OrderUP!~\cite{grimes2010let}, one plays a server at a restaurant where she/he recommends her/his customers the healthy options to keep the job. In LunchTime~\cite{orji2013lunchtime} players can choose a health goal (i.e., manage weight, manage diabetes, manage blood pressure, build muscle, and general well-being) and choose meals from restaurants according to their health goals. More points are awarded if the healthiness of the meals align with the health goal. OrderUP! and LunchTime have shown to increase the players' nutrition knowledge and general feelings for self-efficacy. 

Monster Appetite~\cite{hwang2017monster} is a game that incorporates framed messages encouraging players to 1) make unhealthy choices to feed their avatar or 2) defend healthy choices and focus on the benefits of such choices. A user study comparing the two framed messages showed that when ``healthy'' messages were combined with negative visuals of the avatars, participants were more likely to exhibit healthier snack choices in their simulated snack shop exercise. 



\vspace{-5pt}
\subsubsection{Player Avatar Identification}
Customization can increase user's enjoyment~\cite{birk2016fostering,marathe2011drives,trepte2010avatar,turkay2015effects}, engagement and presence~\cite{ng2013examining}, motivation to play~\cite{turkay2015effects}, and facilitate self-expression in digital environments such as games~\cite{adinolf2011controlling,bailey2009avatar}. Unlike functional customizations (such as changing the difficulty levels of an enemy AI), cosmetic customizations do not affect the player's gameplay but can improve one's self-expression and enjoyment via affecting the appearance of avatars and game objects, and, therefore game environments~\cite{cuthbert2019effects,turkay2014effects}.  

Scholars have posited a human-animal bond in which pet owners feel their pet's psychological and physical health is related to their own well-being~\cite{chesney2007illusion,cohen2001defining,hosey2014human,lin2017exploring,zichermann2011gamification}. Several studies have shown that keeping oneself motivated to stay healthy in order to take care of a virtual pet have succeeded~\cite{ahn2015using,byrne2012caring,lin2006fish,pollak2010s}, which supports the theory on the special bond that humans and animals share. 

Unfortunately, there is a big gap in connecting the avatar identification mechanism to specific learning outcomes, especially in the context of nutrition~\cite{chang2019stereotype,chen2019effects,de2019algebright,lin2019evaluating}. In a simulation game surrounding the 2010 Haitian hurricane~\cite{bachen2016presence}, increased sympathy experienced through playing a role led to female players' interest in learning more about the game topic. Plenty of studies, in this fashion, look into ``learning'' in a broader sense through self-efficacy, intention to learn, increased interest in the topic and context of the game, etc. Though examining nutritional learning outcomes was not a primary goal of our research, we wanted to investigate whether having pet avatars and having their appearances change based on the crowdsourced meal that was fed to them can in turn show potential for nutritional learning, serving the mechanism to create a nutritional teaching tool in the future. 

We incorporated pet avatars as well as both intrinsic and extrinsic motivators~\cite{chen2016scaffolding,habgood2011motivating} in our study to promote user engagement. In our proposed app environment, a user can ``level up'' (progress to the next status) with a high enough meal score, to make their pet avatar happy and healthy. Players can also receive rewards such as hats, toys, and other virtual items (see Figure~\ref{fig:monster-stages}).

From the previous sections, while we see a plethora of studies involving players taking on a role and caring for an avatar as well as those that focus on increasing food and nutritional literacy, there are not many studies where there is a focus on player-avatar-identification/connection and how that may relate to nutritional engagement. Our study attempts to start looking at the relationship of the tamagotchi-style mechanism of taking care of one's avatar and exploring whether that connection can lead to better nutritional engagement and learning. 






