\section{Background and Relevant Work}
The considerable difficulty of correctly estimating nutrition in meals is not limited to the task itself. It is also because resources that can help educate people are usually experts in the field who are costly and inaccessible to many. This has inspired active research in several fields to take the learning into the crowd's hands. 
%Below we describe previous studies examining challenges related to nutritional estimation of meals and efforts for reducing these challenges in several research communities. 


\subsection{Computational and Crowdsourcing Approaches to Nutritional Estimation}
There exist rich literature describing challenges associated with nutritional estimation of meals \cite{berkman2011low,chandon2007obesity,chaudhry2016evaluation,chaudry2013formative,stanton2006nutrition,kim2014energy,lansky1982estimates,schwartz2006ability}. 
% Some of these challenges are related to lack of knowledge as to the exact ingredients and their amounts (especially for meals eaten out, or for those individuals who did not participate in preparing the meals) \cite{chaudhry2016evaluation,chaudry2013formative,stanton2006nutrition}. This lack of knowledge is particularly acute in mixed foods such as sauces and dressings. In addition, many studies showed that both lay individuals and professional dietitians have considerable challenges estimating portion sizes, particularly for large meals \cite{chandon2007obesity}. Moreover, individuals may not be aware of different measurement standards based on how foods are prepared and cooked (e.g., raw veggies vs. cooked veggies~\cite{muller2012ingredient}). Finally, individuals may not be aware of nutritional composition of different foods, such as inclusion of carbohydrates, protein, fat, and fiber \cite{teng2012recipe}. All these challenges led to notoriously inaccurate nutritional estimates, which have consequences in health outcomes~\cite{berkman2011low}, and primary and secondary prevention of diseases~\cite{scott2002health}. 
These challenges present a need for different interventions that may leverage computing, crowdsourcing, and automating that may help alleviate the burden of users when it comes to nutritional assessment and health-tracking. 
%Though tracking food and dietary habits for weight loss, sports training, or management of chronic diseases, is popular, heavy logging and manual data entry with many current apps often lead to reduced engagement and abandonment of these technologies~\cite{cordeiro2015barriers}. As a result, there emerged a host of solutions for \textit{automating nutritional assessment of meals}, thus reducing user burden. Previous solutions for automating dietary tracking typically fall into one of the several general categories. 




% \subsubsection{Structured Recording Mechanisms}
% The first class of these solutions rely on structured recording of meals by selecting meals from nutritional databases or by scanning bar codes in pre-packaged foods. \cite{siek2006we} used an electronic intake monitoring application for patients with chronic kidney diseases to report their food intake through barcode scanning and voice recordings. 
%Only 60\% of the food items scanned through barcodes could be identified, as an open source food database was used and lacked variety of food items that were available at low-cost stores ~\cite{siek2006pride, siek2009evaluation}. 
% These approaches have additional important limitations. Multiple studies showed that selecting meals from nutritional databases often leads to inaccurate records~\cite{schafer2017user}, they tend to be concentrated with commercial and fast food items but lack many ethnic and home-made foods ~\cite{chen2009pfid,zhu2010use}, and barcodes are only available for pre-packaged foods, leading to increase consumption of pre-packaged foods, to simplify data entry–-an undesired side-effect of these technologies ~\cite{alkerwi2015consumption,cordeiro2015barriers}.  
% \subsubsection{Computer Vision and Machine Learning Mechanisms}
% An alternative set of solutions has focused on applying computational image analysis techniques to photographs of meals captured by individuals. 
Some computational image analysis techniques include using machine learning algorithms to recognize components of meals based on training datasets of images of meals labeled by experts or crowd workers. For example, \cite{pouladzadeh2016food} used a convolutional neural network approach in conjunction with a computer vision system to improve food labeling and calorie estimation performance. 
%This team used a labeled dataset of restaurant meals to train a CNN-based multi-classifier that associated depth of each pixel from a single RGB image to the volume of food depicted in the image ~\cite{meyers2015im2calories}. The classifier was then applied to non-restaurant food images to arrive at their caloric estimation~\cite{merler2016snap}. 
Others used 3D shape reconstruction to identify volumes of different foods within identified segments, and use these volumes to arrive at nutritional composition of foods using a nutritional database~\cite{anthimopoulos2015computer,rhyner2016carbohydrate}. 
%However, these approaches continue to be limited in their ability to recognize a wide variety of meals with ingredients that may not be visually apparent~\cite{kong2012dietcam,pouladzadeh2016food,zhang2015snap}; limited to commercial, and/or fast food chain items or food items belonging to a predefined food category~\cite{chen2009pfid,zhu2010use}; or limited in focusing on outcomes on one nutritional aspect such as estimation of calories ~\cite{kuhad2015using,meyers2015im2calories,miyazaki2011image} or portion sizes only ~\cite{chaudhry2016evaluation,chaudry2013formative,nelson1996food}, or one particular population with limited generalizability ~\cite{anthimopoulos2015computer,dehais2016gocarb,dehais2017two}.

% Still with all the automation with meal logging, inherently tracking has a clinical tone to it (REF??). Therefore, alternative approaches, those that are more fun and engaging have emerged to make the tracking process less stiff and clinical. Games are a great way to naturally introduce seemingly heavy and boring topics as nutritional estimation. 

%\subsubsection{Crowdsourcing Mechanisms}
% Previous research also investigated leveraging crowdsourcing and social computing platforms for nutritional estimation. Different combinations of machine learning based or computational strategies with crowdsourced human-centered measurements of foods have been used in many studies to complement each other.
In many previous studies crowdsourcing communities were used to generate labeled datasets to train machine learning algorithms. For example, in the work by \cite{Thomaz:2013:FIE:2526667.2526672} individuals used a wearable necklace camera to take frequent images of their surroundings. All the collected images were then sent to Amazon Mechanical Turkers to be identified as ones related to food to create a database of food images.

Other researchers specifically focused on developing crowdsourcing workflows that help crowd workers arrive at accurate nutritional composition of submitted meals. For example, \cite{noronha2011platemate} pointed that estimating nutrition in meals is a complex task that involves many ingredients and simply asking multiple crowd workers to eyeball nutrition in their meals may not lead to the best results. Instead, these researchers broke nutritional estimation in a number of steps, mimicking typical workflows of trained dietitians and introduced aggregation and voting to resolve conflicts between workers. This approach led to nutritional estimates comparable to those generated by trained dietitians. However, this system struggled to produce good results on liquids like beverages and salad dressings.

One study by \cite{merler2016snap} specifically aimed to create a large data set of ``in-the-wild'' food photos from crowdsourced mechanisms and created one of the largest food databases of real photos from real people. The limitation of this study was that even though these were real people’s meal photos and a sufficiently large database, they were all single food items and food segmentation and portion estimation components were lacking to become useful for dietary assistance and nutritional assessment and evaluation. While these crowdsourced solutions focus on generating accurate estimation of nutrition in meals, they did not specifically examine how to leverage social computing and crowdsourcing to improve nutritional literacy. 

\subsection{Observational Learning Mechanisms for Improvement in Nutritional Literacy}
It is well established that humans can learn vicariously by observing other people's actions and/or decisions and their subsequent consequences. This concept of observational learning is supported when evaluated through the lens of an educational paradigm, `learning-by-example’~\cite{anderson1997role,atkinson2000learning,brown1988preschool} and is also an essential component of Albert Bandura's Social Cognitive Theory~\cite{bandura1989human,bandura1998health}. Observational learning has been utilized in various research fields, specifically for social computing platforms and nutrition. Mamykina et al.~\cite{mamykina2016learning} examined accuracy gains in nutritional assessment of different meals with paid crowd workers when they received expert feedback compared to peer-generated feedback. Expert generated feedback helped in the crowd workers' accuracy gain of nutritional evaluation of meals. However, learning and accuracy gain were also observed when crowd workers could compare their own solutions to solutions provided by others. These findings support the `learning-by-example’ paradigm and how social computing crowdsourced platforms have potential to become an effective teaching tool. 

Burgermaster et al.~\cite{burgermaster2017role} examined gains in nutritional learning with unpaid crowd workers when comparing macronutrient content in pairs of photographed meals (e.g, \textit{Which meal photograph has more carbohydrates?}). The study showed that viewing peer responses or receiving correctness feedback alone did not show any learning gains. However, when explanations were provided with correctness feedback, there were statistically significant learning gains regardless of whether the feedback was provided by experts or peers. Both studies show promise in delivering nutrition content and education through social computing-based casual learning environments. However, these approaches may benefit from adding gamification and gameful components to their platforms, as games have been increasing in popularity and will help motivate the crowd to engage more intrinsically. Our research took these studies and added the gameful element to garner the community's interest in nutritional learning. 

% \textcolor{red}{\textbf{Massive citations of the UDUB studies.} }

% IS THIS WHERE WE SAY WE WILL LOOK INTO THIS ASPECT IN OUR EVALUATION STUDY? MAYBE...


\subsection{Gameful Design and Approaches for Improvement in Nutritional Literacy}
There have been a number of studies that have incorporated game elements in the field of health and specifically nutrition, as games and  gamification in context have shown to encourage players to engage in learning activities and material outside traditional learning environments \cite{alessi1991computer, blumberg2014serious,chen2014healthifying,huizenga2009mobile,jui2011game,mueller2011designing,richards2014beyond}. In addition, gameful design has shown to improve health attitudes and behaviors in a number of studies including nutrition~\cite{deterding2011game,grimes2010let, johnson2016gamification,kyfonidis2019making,peng2009design,orji2013lunchtime,orji2017improving}.



%\subsubsection{Food Literacy and Behavioral Games and Designs}

% Persuasive games are also regarded as a popular tool that have garnered attention as they have shown some positive effects on motivating healthy behaviors \cite{bogost2007persuasive,busch2015personalization,grimes2010let,kato2008video}. 
%Some persuasive games for health promotion and prevention are introduced here. 

Priate Bri's Grocery Adventure incorporated a situated, gameful approach to grocery shopping that could be played at home as well as in grocery stores that sought to improve food literacy through a series of shopping trips~\cite{bomfim2018pirate,bomfim2020food}. The study showed that users who utilized the app were successful in increasing their nutrition knowledge, shopping for healthier food items, and reducing their impulse buys. Their findings suggested that promoting planning for grocery shopping in nutrition apps can be a significant moderator for healthier lifestyles. 

In Escape from Diab~\cite{thompson2008serious}, the main character trains others to increase their physical strength by healthy eating and exercising. Lunch Crunch~\cite{lunchcrunch} makes players place healthy items on lunch trays but trash unhealthy items to educate the players on (un)healthy options. In OrderUP!~\cite{grimes2010let}, one plays a server at a restaurant where she/he recommends her/his customers the healthy options to keep the job. In LunchTime~\cite{orji2013lunchtime} players can choose a health goal (i.e., manage weight, manage diabetes, manage blood pressure, build muscle, and general well-being) and choose meals from restaurants according to their health goals. \textcolor{orange}{Should we say LunchTime was our inspiration?????} More points are awarded if the healthiness of the meals align with the health goal. OrderUP! and LunchTime have shown to increase the players' nutrition knowledge and general feelings for self-efficacy. 

Monster Appetite~\cite{hwang2017monster} is a web-based game that incorporates framed messages encouraging players to 1) make unhealthy choices (therefore called the subversive approach) to feed their avatar or 2) defend healthy choices and focuses on the benefits such choices. A user study comparing the two framed messages showed that when positive messages (the second approach) were combined with negative visuals of the avatars, participants were more likely to exhibit healthier snack choices in their simulated snack shop exercise. 

%\subsubsection{Player Avatar Identification}
% \textcolor{orange}{Definitely need to quote that paper where they mention the tamagotchi style as a game mechanism~\cite{zichermann2011gamification}.And then obviously this serious game about the Haiti's earthquake~\cite{bachen2016presence}. }
% Tamgotchi-style: ``Feed this thing (virtual pet, crop, etc.) regularly or it will die [an example game mechanic p.80]. This approach strongly promotes repeat visits and a sense of accomplishment''~\cite{zichermann2011gamification}

Customization can increase user's enjoyment~\cite{birk2016fostering,marathe2011drives,trepte2010avatar,turkay2015effects}, engagement and presence~\cite{ng2013examining}, motivation to play~\cite{turkay2015effects}, and facilitate self-expression in digital environments such as games~\cite{adinolf2011controlling,bailey2009avatar}. Unlike functional customizations (such as changing the difficulty levels of an enemy AI), cosmetic customizations do not affect the player's gameplay but can improve one's self-expression and enjoyment via affecting the appearance of avatars and game objects, and, therefore game environments~\cite{cuthbert2019effects,turkay2014effects}.  

% ``Avatars as digital representations of players in these synthetic environments facilitate players’ interactions with the game environment and other players. Thus, they are the surrogates
% for users that enhance their embodiment in VR settings. Previous studies found that avatar customisation has a variety of positive impacts on player experiences. These include improved identification with one’s avatar~\cite{birk2016fostering,turkay2014effects}; improved motivation to play~\cite{turkay2015effects}; higher levels of enjoyment~\cite{trepte2010avatar}; and increased engagement and presence~\cite{ng2013examining}.
% ''~\cite{cuthbert2019effects}. 

% ``Results of this study showed strong effects for creating a custom avatar on feelings of engagement and presence, and modest effects on measures of learning.''~\cite{ng2013examining}

Scholars have posited a human-animal bond in which pet owners feel their pet's psychological and physical health is related to their own well-being~\cite{chesney2007illusion,hosey2014human,lin2017exploring,zichermann2011gamification}. Several studies have shown that keeping oneself motivated to stay healthy in order to take care of a virtual pet have succeeded~\cite{ahn2015using,byrne2012caring,lin2006fish,pollak2010s}, which supports the theory on the special bond that humans and animals share. 

We incorporated pet avatars as well as both intrinsic and extrinsic motivators~\cite{chen2016scaffolding,habgood2011motivating} in our study to promote user engagement.
% RELATE THIS ASPECT TO MA??? Maybe the MA thing can be related to choosing your own pet avatar with a specific nutritional/health goal
In our proposed game environment, a player can ``level up'' (progress to the next status) with a high enough meal score, to make their pet avatar happy and healthy. Players can also receive rewards such as hats, toys, and other virtual items. \textcolor{orange}{should have a figure referencing these accessory unlocks but also this whole paragraph may not belong right here. }

From the previous sections, while we see a plethora of studies involving players taking on a role and caring for an avatar as well as those that focus on increasing food and nutritional literacy, there are not many studies where there is a focus on player-avatar-identification and connection and how that may relate to nutritional learning. Our study attempts to start looking at the relationship of the tamagotchi style mechanism of taking care of one's avatar and exploring whether that connection can lead to better nutritional learning. 

%%%%%%%%%%%%%%%% starting here taken from my MA paper for 2017 CHI %%%%%%%%%%%%%%%%%%%%%%%%%%%%%%%%%%%%%%%%%%%%%%%%%%%%%%%%%%%%%%%%%%%%%%%%%%%%%%%%%%%%%%%%%%%%%%%%%%%%%%%%%%%%%%%%%%%%%%%%



% As a matter of fact, games have had a long history of being used as a popular tool in health intervention and prevention programs \cite{baranowski2003squire,baranowski2011video,baranowski2008playing,grimes2010let,pollak2010s,redd1987cognitive,szer1983video}. Some uses of health games with promising results include those regarding cancer and other diseases, where patients have to follow a strict regimen for treatment. In such cases, the game works as an organizational (reminders and alerts) and motivational (gamification  aspects) tool\cite{kato2008video}. Others include video games that work as a distraction from pain and other undesired behaviors \cite{kolko1985effects,o2000evaluation}. 
%and as a tool for training health professionals\cite{graafland2012systematic}. \par



%%%%%%%%%%%%%%%%UP TO THIS POINT FROM my MA paper for 2017 CHI %%%%%%%%%%%%%%%%%%%%%%%%%%%%%%%%%%%%%%%%%%%%%%%%%%%%%%%%%%%%%%%%%%%%%%%%%%%%%%%%%%%%%%%%%%%%%%%%%%%%%%%%%%%%%%%%%%%%%%%%%%%%%%%%%%%%%%%%%%%%%%%%%%%%%%%%%%%%%%%%%%%%%%%%%%%%%%%%%%%%%%%%%%%%%%%%%%%%%%%%%%%%%%%%%%%%%%%%%%%%%%%%%%%%%%%%%%%%%%%%%%%%%%%%%%%%%%%%



% \subsection{Education as a predictor for Nutritional Literacy ?}
% \textcolor{orange}{So @pooja, I have been working on the background all day today, but I am still not sure how I would connect this portion of yours to the rest of the relevant work. Just wanted to give you a heads up. }
% \textcolor{blue}{Pooja: most papers on SES and nutrition focus on dietary intake (which relies on nutrition knowledge but has the added confounded of food price), so looking at education and nutritional knowledge in isolation through our game is certainly an interesting question. Most of these papers group low reduction high school or less, and high education as above HS -- just something to keep in mind as we look for effects. --- and maybe you can find a way to reference the ``Personal Health Oracle'' CHI paper~\cite{desai2019personal} as well as the WISH extended abstract or presentation thing Elliot did(?)~\cite{mitchell2019wish}}


% ``25 articles were selected. The results showed that, in most studies, nutrition knowledge was associated with socioeconomic parameters and eating behaviour...The studies have revealed a greater tendency to assess the relationship of nutrition knowledge with sociodemographic and economic parameters.''~\cite{barbosa2016nutrition}

% ``Higher scores were found among those with lower age, higher personal monthly income, and higher education.''~\cite{vasconcelos2020nutrition}

% ``Individuals with higher education exhibited significantly higher nutrition knowledge (NK) and physical knowledge (PK). Individuals with high fat had significantly higher NK and PK (P < .05) than participants with normal fat percentage.''~\cite{pandit2018knowledge}

% Above high school education was associated with middle and high nutritional knowledge/beliefs ~\cite{beydoun2008nutrition}

% Australian sample -- education independently associated with nutritional knowledge, tertiary (BA) and above is high education~\cite{hendrie2008exploring}


