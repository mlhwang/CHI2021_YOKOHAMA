\section{Conclusion}
We piloted Monster Munch (N=68) as a feasibility-focused study to investigate two major gamification mechanisms: player avatar identification and crowdsourced community board. Though avatars appear frequently as a game mechanic in games and apps with gamification, they are not often paired with crowdsourced social computing platforms and investigated in conjunction with learning. From our pilot study, we witnessed that these mechanisms were not only accepted by the users, but also they encouraged them to engage with macronutrient nutrition topics based on crowdsourced meal photographs. Though the avatar bond from the user was negatively associated with nutrition learning, the pilot highlights these mechanisms as effective tools to further discuss lightweight approaches for nutritional engagement. Future studies should investigate the right combination of gamification mechanisms that can promote nutritional engagement to improve nutritional literacy.


%%
%% The acknowledgments section is defined using the "acks" environment
%% (and NOT an unnumbered section). This ensures the proper
%% identification of the section in the article metadata, and the
%% consistent spelling of the heading.
\begin{acks}
This is partly funded through ABC University's hungry hungry monster grant.
\end{acks}
