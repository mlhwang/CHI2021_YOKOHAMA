%% IMPORTANT NOTICE:
%% For the copyright see the source file.
%% 
%% Any modified versions of this file must be renamed
%% with new filenames distinct from sample-sigplan.tex

%% The first command in your LaTeX source must be the \documentclass command.
\documentclass[manuscript, screen]{acmart}
%%
%% \BibTeX command to typeset BibTeX logo in the docs
\AtBeginDocument{%
  \providecommand\BibTeX{{%
    \normalfont B\kern-0.5em{\scshape i\kern-0.25em b}\kern-0.8em\TeX}}}
    
%% Allows images to be inserted:
\usepackage{graphicx}
\graphicspath{ {./images/} }
\usepackage{wrapfig}

% \usepackage[table]{color}
% \usepackage{multirow}
% \usepackage{array}
% \usepackage{makecell}
% \usepackage{todonotes}
% \usepackage{float}
% \raggedbottom





%% Rights management information.  This information is sent to you
%% when you complete the rights form.  These commands have SAMPLE
%% values in them; it is your responsibility as an author to replace
%% the commands and values with those provided to you when you
%% complete the rights form.
\setcopyright{acmcopyright}
\copyrightyear{2021}
\acmYear{2021}
%\acmDOI{10.1145/1122445.1122456}

%% These commands are for a PROCEEDINGS abstract or paper.
\acmConference[Woodstock '18]{Woodstock '18: ACM Symposium on Neural
  Gaze Detection}{June 03--05, 2018}{Woodstock, NY}
\acmBooktitle{Woodstock '18: ACM Symposium on Neural Gaze Detection,
  June 03--05, 2018, Woodstock, NY}
\acmPrice{15.00}
\acmISBN{978-1-4503-XXXX-X/18/06}

%% Submission ID.
%% Use this when submitting an article to a sponsored event. You'll
%% receive a unique submission ID from the organizers
%% of the event, and this ID should be used as the parameter to this command.
%%\acmSubmissionID{123-A56-BU3}

%% end of the preamble, start of the body of the document source.
\begin{document}
%% The "title" command has an optional parameter,
%% allowing the author to define a "short title" to be used in page headers.
\title{Monster Munch, A Mobile Application for Nutritional Engagement}

%% The "author" command and its associated commands are used to define the authors and their affiliations.
%% Of note is the shared affiliation of the first two authors, and the
%% "authornote" and "authornotemark" commands
%% used to denote shared contribution to the research.
\author{Author1}
\authornote{Both authors contributed equally to this research.}
\email{trovato@corporation.com}
\orcid{1234-5678-9012}
\author{Author2}
\authornotemark[1]
\email{webmaster@marysville-ohio.com}
\affiliation{%
  \institution{Institute for Clarity in Documentation}
  \streetaddress{P.O. Box 1212}
  \city{Dublin}
  \state{Ohio}
  \postcode{43017-6221}
}

\author{Lars Th{\o}rv{\"a}ld}
\affiliation{%
  \institution{The Th{\o}rv{\"a}ld Group}
  \streetaddress{1 Th{\o}rv{\"a}ld Circle}
  \city{Hekla}
  \country{Iceland}}
\email{larst@affiliation.org}

\author{Valerie B\'eranger}
\affiliation{%
  \institution{Inria Paris-Rocquencourt}
  \city{Rocquencourt}
  \country{France}
}

\author{Aparna Patel}
\affiliation{%
 \institution{Rajiv Gandhi University}
 \streetaddress{Rono-Hills}
 \city{Doimukh}
 \state{Arunachal Pradesh}
 \country{India}}

\author{Huifen Chan}
\affiliation{%
  \institution{Tsinghua University}
  \streetaddress{30 Shuangqing Rd}
  \city{Haidian Qu}
  \state{Beijing Shi}
  \country{China}}

\author{Charles Palmer}
\affiliation{%
  \institution{Palmer Research Laboratories}
  \streetaddress{8600 Datapoint Drive}
  \city{San Antonio}
  \state{Texas}
  \postcode{78229}}
\email{cpalmer@prl.com}

\renewcommand{\shortauthors}{Author1 and Author2, et al.}

%%
%% The abstract is a short summary of the work to be presented in the
%% article.


%%
%% The code below is generated by the tool at http://dl.acm.org/ccs.cfm.
%% Please copy and paste the code instead of the example below.
%%
\begin{CCSXML}
<ccs2012>
 <concept>
  <concept_id>10010520.10010553.10010562</concept_id>
  <concept_desc>Computer systems organization~Embedded systems</concept_desc>
  <concept_significance>500</concept_significance>
 </concept>
 <concept>
  <concept_id>10010520.10010575.10010755</concept_id>
  <concept_desc>Computer systems organization~Redundancy</concept_desc>
  <concept_significance>300</concept_significance>
 </concept>
 <concept>
  <concept_id>10010520.10010553.10010554</concept_id>
  <concept_desc>Computer systems organization~Robotics</concept_desc>
  <concept_significance>100</concept_significance>
 </concept>
 <concept>
  <concept_id>10003033.10003083.10003095</concept_id>
  <concept_desc>Networks~Network reliability</concept_desc>
  <concept_significance>100</concept_significance>
 </concept>
</ccs2012>
\end{CCSXML}

\ccsdesc[500]{Computer systems organization~Embedded systems}
\ccsdesc[300]{Computer systems organization~Redundancy}
\ccsdesc{Computer systems organization~Robotics}
\ccsdesc[100]{Networks~Network reliability}

%%
%% Keywords. The author(s) should pick words that accurately describe
%% the work being presented. Separate the keywords with commas.
\keywords{avatars, nutritional literacy, macronutrients, meal photographs, crowdsourcing, community board}


%%
%% This command processes the author and affiliation and title
%% information and builds the first part of the formatted document.
\maketitle


Nutrition is a crucial part of healthy living, however, individuals often struggle to understand and make healthy choices. Past HCI work, particularly in health, has shown the power of gamification in promoting engagement with and understanding of complex information. Leveraging gamification techniques (avatars and crowdsourced feedback) to help people engage with nutrition, we developed Monster Munch, a mobile application where users help monster avatars achieve a particular health goal (e.g., lose weight) by selecting crowdsourced ``in-the-wild'' meals to feed them. We piloted Monster Munch (N=68) and found users' confidence assessing macronutrients increased after using the app. Strong player-avatar-identification (PAID) and increased utilization of the crowdsourced community board were related to users' enjoyment of the app. Interestingly, lower PAID predicted greater recall of nutrition information, independent of prior nutrition knowledge. Findings suggest PAID may be an important mechanism in learning and highlights how fun, lightweight tools can prompt reflection and recall.




  

\section{Introduction}
Unhealthy eating contributes to the obesity epidemic in the United States, which affects 12 million American adolescents and 38\% of adults \cite{trustforamericashealth,cdc2015,cdc2020}. 
%The estimated annual medical cost of obesity exceeded \$147 billion dollars in 2008 [PLACE UPDATED REF]~\cite{ }. 
While there are many factors that contribute to unhealthy eating, past work highlights low nutritional literacy as a key factor, in particular limited ability to accurately estimate nutritional composition (i.e., macronutrient content) of meals ~\cite{kindig2004health}. 
While estimating the nutritional composition of meals is crucial for health management and preventing undesired health consequences (especially for individuals with chronic conditions), previous studies show doing so is a considerable challenge. For example, individuals from low-literacy backgrounds often have difficulty interpreting Nutrition Fact labels, leading to miscalculation of the amount of consumed nutrients \cite{chaudry2013formative,huizinga2009literacy,rothman2006patient}. Even professional dietitians~\cite{chandon2007obesity} and healthy eaters often underestimate calories in meals~\cite{chandon2007biasing}. Complex meals such as salads and dressings with many components add another layer of difficulty with nutritional assessment~\cite{noronha2011platemate}.


Due to the difficulty in nutritional estimation, as well as the huge user burden with many self-tracking and food journaling apps, people develop short-lived commitments to these tools and often are fatigued leading to complete abandonment of the app and self-tracking habits~\cite{choe2014understanding,Cordeiro:2015:RMF:2702123.2702154,cordeiro2015barriers,epstein2016crumbs,mattila2008mobile}. A variety of \textit{interactive, gamified solutions} have emerged to help individuals engage with nutrition information easily, promote meal tracking and behavior change, and improve learning. Noting the high burden of tracking ingredients and meals many users report~\cite{desai2019personal}, some solutions automate the process of nutrition logging 
through bar code scanning and ingredient selection from existing databases ~\cite{beijbom2015menu,bomfim2018pirate,bomfim2020food,siek2009evaluation}, lightweight social interactive challenges~\cite{Cordeiro:2015:RMF:2702123.2702154,epstein2016crumbs},
computational image analysis~\cite{anthimopoulos2015computer,kong2012dietcam,rhyner2016carbohydrate,zhang2015snap,zhu2010use}, crowdsourcing~\cite{mamykina2011examining,noronha2011platemate}, and immersive avatar-based gaming or tamagotchi-style  nurturing~\cite{ahn2017immersive,byrne2012caring,hwang2017monster,lin2006fish}.


While avatars are a commonly utilized gamification mechanism, there is a dearth of literature that has explored the relationship between player-avatar-identification and learning outcomes. A plethora of research has focused around how higher engagement, presence, and enjoyment of and in a game can be facilitated with more options to customize avatars~\cite{ahn2017immersive,bailey2009avatar,birk2016fostering,li2013player,trepte2010avatar,turkay2014effects,turkay2015effects}. Naturally, connecting greater presence and enjoyment in a game to learning is typically the next step in (educational) games research~\cite{de2019algebright,huizenga2009mobile,lin2017character,lin2019evaluating,ng2013examining,vogel2006computer}. However, there are seldom examples where the avatar customization itself and therefore the avatar's appearance embodies a learning concept. In our app, Monster Munch, the pet monster avatars' appearance changes from unhealthy to healthy depending on what meals are fed to them by the user and therefore delivers information about the meal itself and its health consequences. In other words, the appearance of the avatars is intended to do more than incentivize engagement but be an important part of the learning. In addition, crowdsourcing, a typical feature commonly categorized as social, is incorporated into our app to deliver nutritional information through what is called a community board (see Figure~\ref{fig:screenshots}, showing what the crowd fed their monster avatars to achieve the same nutritional goal the present user is trying to achieve for her/his pet avatar.

In this feasibility-focused, pilot study, we tested gamification mechanisms that have received relatively positive user feedback respectively, but have not been mixed together due to challenges in current approaches to nutritional assessment, food journaling, and access to ``in-the-wild'' crowdsourced meal photographs.

Specifically, we developed a mobile application, Monster Munch, where users help a self-selected monster avatar with a particular health goal (e.g., aiming to run a marathon) achieve that health goal by selecting a crowdsourced ``in-the-wild'' meal photograph that fits their monster's needs. When selecting the meal to feed their monster, users viewed meal photos and reviewed the meal selection reasoning of other members of the crowd through a community board (See Figure~\ref{fig:screenshots}). This work examines gamification and social mechanisms in the context of nutrition, closes gaps around engaging people with nutrition information in a simple, fun, crowdsourced, and personalized fashion. Our work examines gamification and social mechanisms in the context of nutrition, taking on the challenge of delivering nutrition literacy in an engaging, fun, crowdsourced, and personalized fashion.

In our pilot study (N=68), we observed that users' self-reported confidence in assessing macronutrients based on meal photographs had increased significantly after using the app. Strong player-avatar-identification (PAID) and increased utilization of the crowdsourced community board (where the crowd had voted and reasoned with what they considered the ``best meal'' of the available options was) were related to users' enjoyment, rating, and recommendation of the app to others. Interestingly, lower PAID predicted greater recall of nutrition information, independent of prior nutrition knowledge. 
%Findings suggest PAID may be an important mechanism in learning and highlights how fun, lightweight tools can prompt reflection and recall.

%\subsection{Contributions}
In summary, the key contributions of this paper are as follows:

(1) This work integrates previously disconnected gamification and social mechanisms (tamagotchi-style mechanism and crowdsourced community board) to facilitate engagement with and deliver nutritional information in a new context;

(2) This paper shows that a lightweight app such as Monster Munch has the potential to engage users and help them improve nutritional literacy and can be established and maintained by the crowd and for the crowd; 

(3) This study emphasizes the role that player-avatar-identification may have in shaping user's engagement and enjoyment of the app and possible effects on recollection and learning outcomes.





\vspace{-5pt}
\section{Background and Relevant Work}
The considerable difficulty of correctly estimating nutrition in meals is not limited to the task itself. It is also because resources that can help educate people are usually experts in the field who are costly and inaccessible to many. This has inspired active research in several fields to automate the process or leverage the crowd's intelligence. 

 
\vspace{-5pt}
\subsection{Computational and Crowdsourcing Approaches to Nutritional Estimation}

There exist rich literature describing challenges associated with nutritional estimation of meals \cite{berkman2011low,chandon2007obesity,chaudhry2016evaluation,chaudry2013formative,stanton2006nutrition,kim2014energy,lansky1982estimates,schwartz2006ability}. These challenges present a need for different interventions that may leverage computing, crowdsourcing, and automating that may help alleviate the burden of users when it comes to nutritional assessment and health-tracking. Some solutions rely on selecting meals from nutritional databases or by scanning bar codes in pre-packaged foods~\cite{siek2006we}. Some computational image analysis techniques include using machine learning algorithms to recognize components of meals based on training datasets of meal images labeled by experts or crowd workers~\cite{anthimopoulos2015computer,pouladzadeh2016food,rhyner2016carbohydrate,Thomaz:2013:FIE:2526667.2526672}. 
Others specifically focused on developing crowdsourcing workflows that help crowd workers arrive at accurate nutritional composition of submitted meals~\cite{noronha2011platemate}. 
 
However, this system struggled to produce good results on liquids like beverages and salad dressings. One study by \cite{merler2016snap} specifically aimed to create a large data set of ``in-the-wild'' food photos from crowdsourced mechanisms and created one of the largest food databases of real photos from real people. \textcolor{red}{TRANSITION SENTENCE}


\vspace{-5pt}
\subsection{Observational Learning Mechanisms for Improvement in Nutritional Literacy}
%It is well established that 
Humans can learn vicariously by observing other people's actions/decisions and their subsequent consequences. This concept of observational learning (OL) is supported through the educational
`learning-by-example’ paradigm~\cite{anderson1997role,atkinson2000learning,brown1988preschool} and Bandura's Social Cognitive Theory~\cite{bandura1989human,bandura1998health}. OL has been utilized in various research fields, specifically for social computing platforms and nutrition. \cite{mamykina2016learning} examined accuracy gains in nutritional assessment of different meals with paid crowd workers when they received expert feedback compared to peer-generated feedback. Expert generated feedback helped in the crowd workers' accuracy gain of nutritional evaluation of meals. However, learning and accuracy gain were also observed when crowd workers could compare their own solutions to those provided by others. These findings support the `learning-by-example’ paradigm and how social computing crowdsourced platforms have potential to become an effective teaching tool. 

\cite{burgermaster2017role} examined gains in nutritional learning with unpaid crowd workers when comparing macronutrient content in pairs of photographed meals. The study showed that viewing peer responses or receiving correctness feedback alone did not show any learning gains. However, when explanations were provided with correctness feedback, there were significant learning gains regardless of whether the feedback was provided by experts or peers. Both studies show promise in delivering nutrition content and education through social computing-based casual learning environments. However, these approaches may benefit from adding gamification and gameful components to their workflow, as gameful design has shown to motivate the crowd to engage more intrinsically. Our research took these studies and added the gameful element to garner the community's interest in nutrition. 

\vspace{-5pt}
\subsection{Gameful Design Approaches for Improvement in Nutritional Literacy}
A number of studies have incorporated game elements in the field of health and specifically nutrition, as games and  gamification in context have shown to encourage players to engage in learning activities and material outside traditional learning environments \cite{alessi1991computer, blumberg2014serious,chen2014healthifying,huizenga2009mobile,jui2011game,mueller2011designing,papastergiou2009exploring,richards2014beyond}. In addition, gameful design has shown to improve health attitudes and behaviors in a number of studies including nutrition~\cite{deterding2011game,grimes2010let, johnson2016gamification,kyfonidis2019making,peng2009design,orji2013lunchtime,orji2017improving}.

Priate Bri's Grocery Adventure incorporated a situated, gameful approach to grocery shopping that sought to improve food literacy through a series of grocery shopping trips~\cite{bomfim2018pirate,bomfim2020food}. The study showed that users who utilized the app were successful in increasing their nutrition knowledge, shopping for healthier food items, and reducing their impulse buys. 

In Escape from Diab~\cite{thompson2008serious}, the main character trains others to increase their physical strength by healthy eating and exercising. Lunch Crunch~\cite{lunchcrunch} makes players place healthy items on lunch trays but trash unhealthy items to educate the players on (un)healthy options. In OrderUP!~\cite{grimes2010let}, one plays a server at a restaurant where she/he recommends her/his customers the healthy options to keep the job. In LunchTime~\cite{orji2013lunchtime} players can choose a health goal (i.e., manage weight, manage diabetes, manage blood pressure, build muscle, and general well-being) and choose meals from restaurants according to their health goals. More points are awarded if the healthiness of the meals align with the health goal. OrderUP! and LunchTime have shown to increase the players' nutrition knowledge and general feelings for self-efficacy. 

Monster Appetite~\cite{hwang2017monster} is a game that incorporates framed messages encouraging players to 1) make unhealthy choices to feed their avatar or 2) defend healthy choices and focus on the benefits of such choices. A user study comparing the two framed messages showed that when ``healthy'' messages were combined with negative visuals of the avatars, participants were more likely to exhibit healthier snack choices in their simulated snack shop exercise. 



\vspace{-5pt}
\subsubsection{Player Avatar Identification}
Customization can increase user's enjoyment~\cite{birk2016fostering,marathe2011drives,trepte2010avatar,turkay2015effects}, engagement and presence~\cite{ng2013examining}, motivation to play~\cite{turkay2015effects}, and facilitate self-expression in digital environments such as games~\cite{adinolf2011controlling,bailey2009avatar}. Unlike functional customizations (such as changing the difficulty levels of an enemy AI), cosmetic customizations do not affect the player's gameplay but can improve one's self-expression and enjoyment via affecting the appearance of avatars and game objects, and, therefore game environments~\cite{cuthbert2019effects,turkay2014effects}.  

Scholars have posited a human-animal bond in which pet owners feel their pet's psychological and physical health is related to their own well-being~\cite{chesney2007illusion,cohen2001defining,hosey2014human,lin2017exploring,zichermann2011gamification}. Several studies have shown that keeping oneself motivated to stay healthy in order to take care of a virtual pet have succeeded~\cite{ahn2015using,byrne2012caring,lin2006fish,pollak2010s}, which supports the theory on the special bond that humans and animals share. 

Unfortunately, there is a big gap in connecting the avatar identification mechanism to specific learning outcomes, especially in the context of nutrition~\cite{chang2019stereotype,chen2019effects,de2019algebright,lin2019evaluating}. In a simulation game surrounding the 2010 Haitian hurricane~\cite{bachen2016presence}, increased sympathy experienced through playing a role led to female players' interest in learning more about the game topic. Plenty of studies, in this fashion, look into ``learning'' in a broader sense through self-efficacy, intention to learn, increased interest in the topic and context of the game, etc. Though examining nutritional learning outcomes was not a primary goal of our research, we wanted to investigate whether having pet avatars and having their appearances change based on the crowdsourced meal that was fed to them can in turn show potential for nutritional learning, serving the mechanism to create a nutritional teaching tool in the future. 

We incorporated pet avatars as well as both intrinsic and extrinsic motivators~\cite{chen2016scaffolding,habgood2011motivating} in our study to promote user engagement. In our proposed app environment, a user can ``level up'' (progress to the next status) with a high enough meal score, to make their pet avatar happy and healthy. Players can also receive rewards such as hats, toys, and other virtual items (see Figure~\ref{fig:monster-stages}).

From the previous sections, while we see a plethora of studies involving players taking on a role and caring for an avatar as well as those that focus on increasing food and nutritional literacy, there are not many studies where there is a focus on player-avatar-identification/connection and how that may relate to nutritional engagement. Our study attempts to start looking at the relationship of the tamagotchi-style mechanism of taking care of one's avatar and exploring whether that connection can lead to better nutritional engagement and learning. 







\vspace{-5pt}
\section{Implementation}
In this section, we outline the processes of preparing the tools we used in our study, including the survey elements of the pre-app task and post-app task, as well as the app itself.
\vspace{-5pt}
\subsection{Development of the App}
The app was created using the visual programming interface, MIT App Inventor 2, version nb185a, and exported as a Android Package Kit (APK) using MIT App Inventor Code, version code36. An APK is a file format used by Android to distribute and install apps. The app was programmed to collect detailed logs of users' interactions (e.g., clicks, time-stamps) to assist in identifying user behavior and level of interaction. 


This app begins by presenting monster avatars that each have an associated macronutrient goal. In each round users are shown four meal photos and are asked to pick which one best fits their monster's goal then to type in a short reason why they selected that meal for their avatar. Once a meal photo is selected, users are taken to a community board (CB) to see what others decided. Each user has the option of changing their mind or moving forward with their original selection. Once more the user types in the reason for sticking to their choice or changing their meal. Once a meal photo is selected the round is over. There are a total of five rounds. The app also randomized the meal choices within each round, as well as the order of the rounds themselves. App components are further described in more detail below (see Figure~\ref{fig:screenshots}).



\begin{figure}[!ht]
\includegraphics[scale=0.33]{samples/images/figure-2-02.png}
\caption{Chart showing the steps of the study and the differences between the Data Collection Phase and Pilot Phase.}
%\vspace{-15pt}
\label{fig:phasechart}
\end{figure}
%\setlength{\textfloatsep}{3pt plus 1.0pt minus 2.0pt}


\begin{figure}[!ht]
\includegraphics[width=\textwidth]{samples/images/screenshots2.png}
\caption{Series of screenshots showing the main steps of the app. Pilot Phase only. }
\label{fig:screenshots}
\end{figure}


\vspace{-5pt}
\subsection{Gamification Mechanisms for the App}
\subsubsection{The Ability for the User to Select and Name a Pet Monster Avatar} 
There are four monsters in the app, each with their own brief background story as to why they are pursuing their nutritional goal. 
For example, one monster's story reads: \textit{``This monster has Irritable Bowel Syndrome. Their daily life is greatly influenced by the way their digestive system behaves. One part of their treatment plan is managing their diet. Their dietitian has given them the goal of increasing the amount of fiber in their diet''}. 

In the Data Collection Phase (DCP), participants were assigned to their pet monster avatar and provided with its story and corresponding nutritional goal. In the Pilot Phase (PP), users could choose their own monster avatar with a corresponding health goal. After selecting one of the four monsters, users were asked to give their monster a name. 
This custom name was used throughout the rest of the app. Figure~\ref{fig:screenshots} shows screenshots of the flow.


\vspace{-5pt}
\subsubsection{Viewing the Community Board (CB) of Crowdsourced Intelligence}

The inclusion of a CB made input from the crowd available to assist the user in deciding which meal best fits the chosen nutritional goal. 
The CB displayed both the percentage of the crowd that selected each of the four meal options and three user-generated reasons explaining why users chose that meal option. The user-generated reasons presented in the pilot were collected in the DCP. 




\vspace{-5pt}
\subsubsection{Monster's State Change and  Unlocking Accessories} 

After users viewed the CB and submitted their final meal choice,
%with a text-based reason, 
they were able to watch their pet monster avatar react to the meal they chose to feed them. 
If the monster was fed the best choice meal for their nutritional goal, the user saw their monster avatar's physical state improve in a short animated morph. If the monster was given the second best meal, their condition remained unchanged. 
If the user fed their pet monster either of the two worst options, they saw its physical state degrade in the animated morph. If within the five meal rounds, a user answered four or more questions with the best meal option, the user unlocked the accessories screen, where they were able to choose one of four accessories to award their pet monster as shown in Figure~\ref{fig:monster-stages}. 
Given the five rounds, users could win up to two accessories.

\begin{figure}[h]
\includegraphics[scale=0.25]{samples/images/figure-5.png}
\caption{Sample of various monster condition stages. Monsters started in the middle, ``Neutral,'' and became ``Less Healthy'' (left) or ``More Healthy'' depending on user's meal choices. The far right shows avatars with the available accessories when unlocked.}
\label{fig:monster-stages}
\end{figure}



\vspace{-5pt}
\subsection{Development of Surveys}
There were two pre-task surveys: 1) a demographic questionnaire, and 2) pre-test nutritional assessment.
In the demographic questionnaire, users were asked basic demographic questions and questions about their prior experiences with nutritional knowledge and health related apps. The pre-test nutritional assessment consisted of 12 questions in which users were tasked with identifying which of two meal options better fit a given macronutrient content goal (e.g., which meal photograph is higher in carbohydrates). 
Participants did not receive feedback on the accuracy of their responses. 
There were four questions for each macronutrient type (i.e., carbs, fat, fiber). 

The post-task consisted of three surveys: 1) the post-test assessment, 2) player-avatar-identification (PAID) questions, and 3) user experience questions.
The 18 question post-test assessment was similar to the pre-test assessment and also asked the user to identify which of two meal options better fit for the given macronutrient goal.
The 18 meal identification questions consisted of six questions for each of the three macronutrient types. For each macronutrient type, four of the six questions contained meals repeated from the pre-test assessment, with the remaining two questions based on new meals. No meals from the app were used in the pre- and post-test assessments. Participants did not receive feedback on the accuracy of their responses. 

The PAID questions were adapted from the previously validated Player-Avatar Identification Scale~\cite{li2013player}. %using a 5-point Likert scale. 
The user experience questions focused on why users chose their monster, since users were able to select their pet avatar. The consent form, demographic questionnaire, pre-test assessment, post-test assessment, player-avatar identification questions, and the user experience questions were all created and collected using the online survey platform, Qualtrics.

\vspace{-5pt}
\subsection{Meal Photo Selection}
The app features 37 photographs of ``in-the-wild'' meals, and the pre- and post-test assessments used an additional 36.
The 73 total meal photographs were gathered from the researchers' prior studies
%~\cite{desai2019personal,mitchell2019wish} 
that used professional dietitians to complete the expert nutritional assessments. We filtered the images based on their resolution quality and content clarity, so that users were able to easily identify the components of the meals.

In each of the five rounds of the app, users were presented with four ``in-the-wild'' meal photographs, each accompanied by a brief description of its contents (e.g., Fat free Greek yogurt with grapes and coffee). We created the rounds so that each had three types of outcomes: one best choice, one second best choice, and two equally less desirable choices, ranked as such by their macronutrient content assessment. We tested the accuracy of the meal choice rankings by circulating all rounds to nutrition experts. 
We then adjusted the questions as necessary to achieve above 50\% expert agreement for all rounds and assessments. 

In the pre- and post-test assessments, each question compared two ``in-the-wild'' meal photographs, with only one being the correct option, and the other being incorrect.

\vspace{-5pt}
\section{Monster Munch Pilot Study}
Because the Data Collection Phase was used to populate data in the community board, and therefore the creation of the app, the report of the users from the collection phase will not be mentioned moving forward. See Figure~\ref{fig:phasechart} for details.

%The pilot with Monster Munch was conducted with participants recruited via both Amazon Mechanical Turk (AMT) and social media sites. 
The goals of the pilot study were to (1) to assess users general perceptions using a lightweight app for engaging with nutrition and (2) understand how the different gamification and social mechanisms built into the app (avatars and crowdsourced community board) shape users preferences and experience with the app. The pilot study followed the procedure outlined below and summarized in Figure~\ref{fig:studyflow}. 


\begin{figure}[ht]
\includegraphics[width=\textwidth]{samples/images/figure-1.png}
\caption{Study flow for participants in the Monster Munch pilot study. Study sections indicated inside the curly braces were completed within the mobile app. }
\label{fig:studyflow}
\end{figure}

\vspace{-5pt}
\subsection{Participant Recruitment}
Participants were recruited from Amazon Mechanical Turk (AMT) and social media sites (Facebook and Instagram). Users were screened with the following inclusion criteria: 1) own an Android mobile device, 2) above 18 years of age. After accepting the terms of the study found in the consent form, participants  downloaded an APK file and installed the Monster Munch app. Within the app they completed (1) a pre-test evaluation, (2) Monster Munch nutrition activities, and (3) post-test assessment and surveys. Only participants who completed all three tasks were considered in subsequent analyses. 


\vspace{-5pt}
\subsection{Monster Munch App Activities}

In the Monster Munch app, users could view each of the four pet monster avatars and their nutritional goals before selecting one to proceed with and ``help'' during the app. Users also had the opportunity to give their monster a custom name before completing the task and helping feed their monster.

In the core monster feeding task, users selected meals to feed their chosen monster for five rounds. Each round of meal selection consisted of five steps. 
First, the user was presented with  four options of ``in-the-wild'' meal photographs, accompanied with brief descriptions of the contents of each meal that they could review to decide what to feed their monster. Second, users were asked to select the meal that they believed best fit their monster’s nutritional goal and provide a short text-based description of why they selected that particular option. Third, users viewed the crowdsourced community board (CB) where they could consider which meal other members of the community chose to feed their monster. Users could view the percent of users that selected each meal and their reasoning for doing so (these percents and rationales were sourced from the data collection phase described above). Fourth, armed with this new information, users had the option to keep their original meal selection, or switch to another meal option. Again, users were asked to provide a short description to rationalize their final meal selection after viewing the input from the CB.
Finally, after they submitted their reasoning, the app told the user whether the meal option they selected was (in)correct, and which meal option was the best choice. The appearance of the monster avatar also changed in response to the user's performance, becoming ``healthier'' if the user selected the correct meal, and ``less healthy'' if the user selected the incorrect meal. 
Users repeated these steps for five meal selection rounds with the same  monster avatar, helping them work towards their monster's nutritional goal.
\vspace{-5pt}
\subsection{Data Analysis and Measures of Interest}
Participants' survey responses were collected via Qualtrics and their engagement with the app  collected via the Google Cloud computing platform Firebase and extracted with Python 2.7. Users' data from Qualtrics and Firebase were merged together for processing, and 
%\url{https://github.com/mlhwang/m4m}
Statistical Package for the Social Sciences (SPSS version 27) was used to analyze the data. For all analysis the alpha significance threshold was set at 0.05.  

\subsubsection{Player Avatar Identification}
Users' engagement with the gamified avatar component of the task was assessed with four Player Avatar Identification (PAID) questions asked in the post-task survey (adapted from~\cite{li2013player}). Users indicated their agreement to the following four statements on a 7-point Likert scale (with the option to select ``Not Applicable''): 
\begin{enumerate}
    \item ``When my monster’s condition worsened, I felt angry/sad.''
    \item ``When my monster achieved their goals, I felt happy.''
    \item ``My monster reflects who I am.''
    \item ``My monster influences the way I feel about myself.'' 
\end{enumerate}
A ``PAID score'' was calculated for each user by summing their responses to each question. Higher PAID scores indexed stronger player-avatar identification.

\subsubsection{Community Board (CB) Agreement}

The CB was created through the Data Collection Phase. The two CB related questions during the post-app task included: 
\begin{enumerate}
    \item ``Did the community board influence your final meal choice?'' 
    \item ``How often did you agree with the rest of the community?'
\end{enumerate}

Users responded on a scale of `Always;' `More than half of the time;' `About half of the time;' `Less than half of the time;' `Never;' and `I did not look at the community board.' 

\subsubsection{Macronutrient Knowledge Assessment}

Macronutrient knowledge was evaluated before and after the Monster Munch app activities. The percentage correct was calculated for the 12-question pre-test to establish a baseline for nutritional knowledge. After using the Monster Munch app, users completed an 18-question post-test, 12 repeated-questions and six new questions. Performance on the 12 repeated questions was used to assess recall of nutrition knowledge, and performance on the six new questions was used to assess transfer of nutrition knowledge in comparison to the baseline evaluation. 
\vspace{-5pt}
\subsection{Research Questions}
In the next section we discuss the results from this pilot study using Monster Munch. In particular this pilot evaluation was centered around understanding the following questions:
\begin{enumerate}
    \item Does playing Monster Munch influence users' confidence of their ability to estimate macronutrient content of ``in-the-wild'' meal photographs (a proxy for nutritional engagement)?
    \item Does the inclusion of gamification and social mechanisms (pet monster avatars and crowdsourced community board) influence enjoyment/engagement of the Monster Munch app?
    \item Do gamification and social mechanisms have an effect on nutritional learning?
\end{enumerate}

% The hypotheses are:
% \begin{itemize}
%     \item \textbf{H1}: User's will report higher enjoyment and show greater engagement with the Monster Munch app after the inclusion of Gamification Mechanisms. 
%     \item \textbf{H2}: Monster Munch (a mobile app for nutritional engagement) will help users become more confident in their ability to assess macronutrient content of ``in-the-wild'' meal photographs pre- to post-test.
%     \item \textbf{H3}: Users who chose pet avatars with a health goal that is personally meaningful in Monster Munch are more likely to perform better (i.e., identify the right meal photographs for a specific nutritional goal) pre- to post-test.

% \end{itemize}

 




\section{Results}
%\subsection{Preliminary Results}
We piloted Monster Munch (N=68) and found users' confidence assessing macronutrients increased after using the app. Strong player-avatar-identification (PAID) and increased utilization of the crowdsourced community board were related to users' enjoyment of the app. Interestingly, lower PAID predicted greater recall of nutrition information, independent of prior nutrition knowledge. Findings suggest PAID may be an important mechanism in learning and highlights how fun, lightweight tools can prompt reflection and recall.





%%%%%% from my MA study
Statistical Package for the Social Sciences (SPSS version 27) was used to analyze the data and alpha was set at 0.05. 
%The first step analyzed participants' demographic information. A chi-square test of independence was performed to compare the frequency in the age ranges in the two-sided inoculation (\textit{n} = 120) and subversive (\textit{n} = 105) conditions. The relation between these variables was not significant, $(\chi^2 (5) = 2.996, p = .701)$. A chi square test of independence was performed to see if there were any significant relationships within gender, education level, income level, or prior game experiences between the two treatment groups. No statistical significance was found between the two groups in their gender $(\chi^2 (2) = 1.222, p = .543)$, education level $(\chi^2 (3) = 3.267, p = .352)$, income level $(\chi^2 (4) = 5.235, p = .264)$, and prior game experience $(\chi^2 (4) = 4.702, p = .319)$ as shown in Table~\ref{tab:learning}.''

%%%%%%%%%%%%%%%%
%%%%%%%%% comment out below for first draft

% This will be the github URL that we will share in the paper for the final draft. Not for anonymized version: \url{https://github.com/mlhwang/m4m} --- for preparing
% high-quality tables.

%%%%%%%%%%%%%%%%%%%%%%%%%%%%%%%

\subsection{confidence assessing macronutrients}

\begin{table}
  \caption{Frequency of Special Characters}
  \label{tab:freq}
  \begin{tabular}{ccl}
    \toprule
    Non-English or Math&Frequency&Comments\\
    \midrule
    \O & 1 in 1,000& For Swedish names\\
    % $\pi$ & 1 in 5& Common in math\\
    % \$ & 4 in 5 & Used in business\\
    % $\Psi^2_1$ & 1 in 40,000& Unexplained usage\\
  \bottomrule
\end{tabular}
\end{table}


\begin{table*}[t]
  \caption{The pre- and post-task means paired samples \textit{t}-test of identifying macronutrients (i.e., fat, carbohydrates, fiber) from meal photographs. \textsuperscript{***}$p<.01$, 
  \textsuperscript{**}$p<.05$, 
  \textsuperscript{*}$p<.1$}
  \label{tab:learning}
  \begin{tabular}{ c  l  r  c  r  c }
    \toprule
    Pair&N&Mean&Std. Dev&t&Sig. (2-sided)
    % \multicolumn{1}{|c|}{\textbf{Pair}}
    % & \multicolumn{1}{|c|}{\textbf {N}} 
    % & \multicolumn{1}{|c|}{\textbf {Mean}} 
    % & \multicolumn{1}{|c|}{\textbf {Std. Dev.}} 
    % & \multicolumn{1}{|c|}{\textbf {t}}
    % & \multicolumn{1}{|c|}{\textbf {Sig. (2-sided)}}\\
    \midrule                                  
    Confidence Pre to Post  &67 &-6.075  &14.274 &-3.484   &0.001**\\
    \bottomrule
\addlinespace[1ex]
\end{tabular}
  \vspace*{-\baselineskip}
\end{table*}




\subsection{nutritional learning }

\subsection{player avatar identification and community board}


SAMPLE OF REPORTING INDEPENDENT T-TESTS


``There was a statistically significant difference between pre- and post-gameplay PDQ means in Session 1 (\textit{M} = .386, \textit{SD} = 1.24, \textit{t}(56) = 2.36, \textit{p} = .022) but not in Session 2 in the subversive group.'' \textcolor{orange}{Remember in this example reporting EFFECT SIZE is not included. We must report that.}



SAMPLE TABLE TO summarize numeric data and indicate p values. \textcolor{orange}{Maybe it does look better to create the Tables without borderlines on the sides so they look more open?}




\textcolor{orange}{Take from \cite{cuthbert2019effects}:THIS SHOULD BE A GOOD MODEL FOR REPORTING ANOVAs if we use those stats techniques.   $\eta_{\text{p}}^{2}= some number to report partial eta square$.}


To set a wider table, which takes up the whole width of the page's
live area, use the environment \textbf{table*} to 
desirable. Immediately following this sentence is the point at which
Table~\ref{tab:commands} is included in the input file; again, 

\begin{table*}
  \caption{Some Typical Commands}
  \label{tab:commands}
  \begin{tabular}{ccl}
    \toprule
    Command &A Number & Comments\\
    \midrule
    \texttt{{\char'134}author} & 100& Author \\
    \texttt{{\char'134}table}& 300 & For tables\\
    \texttt{{\char'134}table*}& 400& For wider tables\\
    \bottomrule
  \end{tabular}
\end{table*}

Always use midrule to separate table header rows from data rows, and
use it only for this purpose. 



\subsection{Figures}

The ``\verb|figure|'' environment should be used for figures.
\begin{figure}[h]
  \centering
  \includegraphics[width=\linewidth]{template_files/sample-franklin}
  \caption{1907 Franklin Model D roadster. Photograph by Harris \&
    Ewing, Inc. [Public domain], via Wikimedia
    Commons. (\url{https://goo.gl/VLCRBB}).}
  \Description{A woman and a girl in white dresses sit in an open car.}
\end{figure}

Your figures should contain a caption which describes the figure to
the reader.

Figure captions are placed {\itshape below} the figure.



A figure description must be unformatted plain text less than 2000
characters long (including spaces).  {\bfseries Figure descriptions
  should not repeat the figure caption – their purpose is to capture
  important information that is not already provided in the caption or
  the main text of the paper.} 

\subsection{The ``Teaser Figure''}

A ``teaser figure'' is an image, or set of images in one figure, that
are placed after all author and affiliation information, and before
the body of the article, spanning the page. If you wish to have such a
figure in your article, place the command immediately before the
\verb|\maketitle| command:
\begin{verbatim}
  \begin{teaserfigure}
    \includegraphics[width=\textwidth]{samples/template_files/sampleteaser}
    \caption{figure caption}
    \Description{figure description}
  \end{teaserfigure}
\end{verbatim}

%, particularly with macronutrient assessment based on crowdsourced meal photographs. 
% and found users' confidence assessing macronutrients increased after using the app. Strong player-avatar-identification (PAID) and increased utilization of the crowdsourced community board were related to users' enjoyment of the app. Interestingly, lower PAID predicted greater recall of nutrition information, independent of prior nutrition knowledge. Findings suggest PAID may be an important mechanism in learning and highlights how fun, lightweight tools can prompt reflection and recall.

\vspace{-5pt}
\section{Discussion}

% What are the questions we asked
We piloted Monster Munch (N=68) as a feasibility-focused study to test out two main mechanisms: player avatar relationships and a crowdsourced community board as a resource for decision making in the app. Though avatars are frequently utilized gamified apps and games, there are few instances of them embodying learning content and paired with crowdsourced social features. Beyond understanding the impact on enjoyment of the app that users experienced with these these mechanisms, we wanted to understand the level to which these mechanisms encouraged users to engage with nutrition topics - a notoriously difficult task in the space of nutritional education.

% concisely highlight key findings here - what do these findings mean, what are you about to argue below?
We found that overall our users enjoyed using the app and their self-reported confidence for assessing macronutrient content in meal photographs increased post app-usage.
A key insight was that specifically, identification with the pet monster avatar, as well as engagement with the community board, was highly predictive of enjoyment, rating of the app, and recommending the app to others. 
Learning was only moderately shown with those who did not identify with their avatars highly, showing that further investigation is due for measuring learning outcomes as a by-product of exploring gamification and social mechanisms.  


\vspace{-5pt}
\subsection{Gamification Mechanisms with User Experience and Learning}
Overall, we found positive experiences reported by the users with their pet monster avatars. The Player-Avatar-Identification (PAID) score was a strong predictor of enjoyment, rating, and recommendation of the app to others. In the post-task, we asked users to rate how much they liked using monsters to learn about meals and nutrition and majority of the users expressed \textcolor{red}{report STATS} liking the monster avatars a great deal as a vehicle for delivering meal correctness feedback. In particular, participants who reported having a high school education or less identified significantly more with their avatars, possibly because it is generally more accessible to retrieve nutritional information through visual effects as opposed to text. In addition, we purposefully chose monster avatars with the intention of eliminating additional complexity that can be derived from various aspects such as gender influence. This aspect probably aided in terms of receiving majority positive feedback regarding the pet monster avatars.  

In our study, users were allowed to choose a monster avatar with specific health goals \textcolor{red}{(REALIZED WE DIDN'T LOOK INTO THIS PARTICULAR QUESTION ---Let's write in some ``common'' reasons why people have chose their specific monsters and I think that might help this paragraph.)} and then were allowed to name their pet avatar. It is possible that these features helped the users feel more associated with the characters by taking a form of ``ownership'' of the avatar. Fondness for a character (Cohen, 1999), attractiveness of avatars (Kim, Lee, \& Kang, 2012), the capabilities of the character (Newman, 2002) are some of the features that have been identified in facilitating avatar identification~\cite{turkay2014effects}. Though the interaction time with the pet monster avatar in the app was not very long (AVERAGE TIME SPENT ON APP FOR ALL --- ADD HERE), many users have mentioned how the monster avatars were cute and adorable in the free feedback question regarding the entire study, not even specifically asking the user's perception of the avatars, which supports the positive evaluation towards the avatars. One aspect that was unfortunate of the app was that most users did not become aware of the ``unlocking accessories'' feature, unless they chose the correct meal at least four times in a row. Typically, cosmetic avatar customization can be strongly related to identification~\cite{turkay2014effects}, but our users did not know about the accessories. They did, however, ``controlled'' how the avatars would look by selecting the correct or incorrect meal to feed them. Having more options for avatar customization that users were aware of should be considered in the future. 

%%%%%%%%%%%%%%%%%%%%%%%%%%%% LEARNING %%%%%%%%%%%%%

We were surprised to find that lower PAID scores marginally predicted greater recall of nutrition information at post-test, even after controlling for baseline nutrition knowledge, educational background, and Monster Munch app performance. It is possible that the high identification with the pet monster avatars and the crowdsourced-based community board were excessive inputs and stimuli for our users. One study~\cite{makransky2019adding} showed that placing university students in an immersive virtual science world through VR headsets led them to higher presence in the virtual learning simulation but resulted in poorer learning outcomes and higher cognitive loads. Our users had to process quite a bit of information in each round in the app: visually they had to view the four individual meal options; they had to read the descriptions of those meals and process that information; they had to write a text-based reasoning for their initial choice to feed the pet avatar; they had to then process the information in the community board that included voting and reasoning for all four meal options; and then they had to provide an additional text-based reasoning for their final choice to feed their monsters. Only then they were given correctness feedback of the meal they decided to feed their monster avatar. In addition, while the app always gave four meal options to choose from, the pre- and post-assessment questions were always a choice between two meal options and correctness feedback was never provided, which does complicate the definition of ``recall learning'' that we refer to in this section. Further investigation is necessary to measure specific learning outcomes that may mimic the intervention activities more closely to see if avatar identification emerges as an important possible predictor of nutritional knowledge and literacy.

% It is possible that the correlation we are seeing between higher PAID scores and lower learning scores suggests that the users that related more to their avatars with nutritional goals because they themselves struggle with nutrition? 

% This would suggest that the user would have higher engagement with an extended version of the Monster Munch app, leading to a future increased nutritional learning?

Though not directly related to learning outcomes per se, our study showed that users' self-reported confidence of their ability to assess macronutrients increased after using the app. This shows great potential for photo-based macronutrient nutrition education, as confidence in learning is shown to be correlated to actual learning outcomes ~\cite{badura1995exercise,choi2005self,de2013relationship,nicholson2013key,wesson2011self}. One user mentioned that this app has great potential as an education app (ID NUMBER) and another user said: \textit{``If people care about their nutritional intake, it is a cute [app]! I am a vegetarian so not the biggest fan about seeing many pictures of meat. ''} (shloming ID).
This shows that there is potential to improve the current app to cater meal photos to individual's dietary preferences and restrictions and this could further aid in the app becoming a more effective educational, teaching tool. 

% -- in general, though it was not required, majority wrote a positive response about the study (most were short like , fun, good, awesome, but all still positive and question was not required to input anything)
% \textit{``it seems interesting, best of luck! I would definitely use this app.''} (ID-043845)

These results suggest the affinity between players and their avatars may be a crucial component to the overall positive experience with the app.

We saw that user's self-reported frequency of using the community board's suggestion (to determine what to feed their monster avatar)  significantly predicted their enjoyment, rating, and recommendation of the app. 

Together these findings highlight the importance of user's engagement with gamification and social components to their overall experience using and enjoying the app. 

% Theory of behavior change --- engagement is one of the first steps of this theory
% learning takes a long time so they behavioral change , so we are looking for more active learning , how do you attain sustainment, or engagement, continue users to inquire about nutrition topics, well how do you get that? with ENGAGEMENT


% ``Self Determination Theory (SDT, Deci \&
% Ryan, 1985) posits that autonomy, competence
% and relatedness are necessary for people’s wellbeing, and was used previously to explain motivational aspects of MMOs (Przybylski, Rigby
% \& Ryan, 2010).''~\cite{turkay2014effects}


% `` previous studies determined
% various player behaviors, features of avatars, and
% characteristics of virtual worlds, that facilitate
% identification. Among these features are fondness for a character (Cohen, 1999), attractiveness of avatars (Kim, Lee, & Kang, 2012), the
% capabilities of the character (Newman, 2002),
% point-of-view (Lim & Reeves, 2009), and
% physical resemblance of avatars to their users
% in body shape, race, age, and facial features
% (Maccoby & Wilson, 1957; Williams, 2011).''~\cite{turkay2014effects}


% `` because players chose
% their avatars’ aspects they felt more associated
% with the character by “taking ownership” of
% it.''





\subsection{Limitations and Future Work}

% Limitations of this research study in this particular scope is discussed in the following sections. 

% \textcolor{red}{What about > While we asked ``What do you think of using monster avatars and their nutritional goal as a way of assessing macronutrient content in meal photographs?'' as an open-ended question in the Data Collection Phase, we had decided to reformat the question on a scale for the Pilot Phase (PP) in an attempt to lessen the effort? asked of the participant as the neared the end of the study.} In the DCP responses for this question, we received a few interesting insights: ``I think it is silly, more for kids than adults''; ``It seems like it would attract peoples' attention a little more than the alternative of just discussing nutrition in a totally abstract way.''; ``It creates a personal feeling which may be advantageous to getting better results''; ``It adds a fun and lighthearted aspect to learning about these things.''; ``It's something that i've never seen before, it caught my attention and i believe it will catch The attention of other People too.''; ``Monster avtar is great and it has sharpen my knowledge that which food has more carbs.'' Among the people who contributed to this question (STATS), certain \% was positive. 

% --- even though the negative comment about the app being for kids and silly is from DCP, this may be an explanation why there was a low PAID score with higher educated people. Maybe the perception was that, the monsters were less relateble because it seemed a bit childish for them. 

% \subsubsection{Limitations with the Gamification Mechanisms}

For the ``unlock accessory'' feature, players were not made aware accessories were available to ``upgrade'' one's monster avatar if four and five rounds worth of meals were correctly fed to the avatar. Only those who reached the four and five correct answers in a row discovered the ``unlock accessory'' feature. If the feature was announced to every user from the beginning and that this could be a goal to strive for maybe we would have witnessed longer duration time stamps for the app activities, and more careful choices might have been made. Additionally, there was no final ``winning'' or ``losing'' status mentioned to the players. If we made receiving accessories the ``winning state'' for the users, the app would have felt more like a game and could have fostered more motivation to play longer as well. We were considering an option to add a button to ``play again'' with a different or the same monster and measured their willingness to continue ``playing the game.'' This would be an excellent next phase to measure engagement and enjoyment in more dynamic ways. 

% Given the easy nature of data collection, Amazon Mechanical Turkers (AMTs) were used for the data collection phase and half of the pilot participants were AMTs. We did receive quite a variety of responses with the AMTs, and in particular, when they were asked to provide a text-based reason for their meal choices before the community board (CB) and after CB, it became clear that many were not reading the instructions or did not put much effort into the task. We rejected a number of HITs as those responses were of ``low-quality'' from our research perspective. Whether we rejected and approved the ``right'' responses will be difficult to assess in any capacity. 

% How much can we guarantee that the post-test responses were not influenced by fatigue. (Maybe this is a good place to quickly mention the average duration people took to complete the entire study? -- first mention this in the Evaluation section when maybe talking about Participants)


% \subsubsection{Determining the Gold Standard for ``in-the-wild'' Meal Photographs}
% First of all, we involved quite a number of people to achieve this. But also the meal photographs are from our previous studies~\cite{} -- keep it anonymized-- and that means most meals are from a particular geographical population (Washington Heights). In creating a true ``crowdsourced intelligence'' driven community board we would have to do a much more comprehensive collection of meal photographs. Maybe using soical media sites like IG and TikTok might help?

% Taken from \cite{cuthbert2019effects}:
% ``Secondly, this study had a small number of participants
% which made the statistics very vulnerable to type II errors.
% We tried to deal with this limitation by reporting effect sizes,
% descriptive statistics, and all the p values, avoiding the typical
% practice of sharing only statistically significant results.''


Because this was a pilot study, the true form of what we want to produce is far from what we tested here. In fact, though we would love a full-fledged game (we did not quite make it a game and only tested out a few gamification mechanisms) freely used by the crowd and the content driven by the crowd, in this particular pilot, we had to pre-process a lot of the data ourselves. The meal photographs were from real users of an app developed in our previous work (keep anonymous) but the gold standard for their evaluation were taken from professional dietitians as we needed a starting point. In future iterations of this study, we will use the data collected in our pilot and include the voting and reasoning for meal choices in the community board. In this way the community board will be created, maintained, and updated truly by the crowd's input and opinion.  
%(\textcolor{orange}{why not use the users' evaluations instead? What is our rationale for this? if we wanted a crowdsourced CB in the truest sense, we should have taken the users' assessments. so why?}

For meal options that were clearly not preferred and fitting the nutritional goal of the monster avatar, we had no text-based reasoning/descriptions for why that particular meal would be the ``right'' choice. Therefore, we had to request our users in the Data Collection Phase to assume that those meals were the right meals for the nutritional goal and provide text-based reasoning for why the meals fit a particular nutritional goal. Moving forward meals that completely do not fit a nutritional goal, and therefore, were not voted for by the crowd should be left empty with no text-based reasoning. If it is clearly not the right choice by the crowd, then let that evaluation show in the community board.  
%(\textcolor{orange}{Again, why not use the users' true evaluations instead? As in there were ZERO reasons for a meal to fit a nutritional goal. What is our rationale?}


All APKs and relevant data pre-processing and data extraction scripts are available on our Github for the HCI community (anonymous link here). 



% How do we get to a truly crowdsourced community board that really represents all people and all types of food? Is that even possible? Could there be one place, one repository for this, or should there not be just one repository? How can we do this more at scale?

\textcolor{orange}{This app with gamification and social mechanisms is a platform that has a potential to grow and aggregate many more crowdsourced, ``in-the-wild'' meal photographs that are also evaluated by the crowd. Given that we involved so many experts, and the gold standard for the meals used were also from trained dietitians, and it took us considerable amount of time to filter through the meals, this whole lightweight and ``naturally crowdsourced'' claim may be difficult. So how do we avoid that attack from the reviewers?????????}

Next step is to make this into a full-fledged game, with winning and losing states, have the crowdsourced community add more meal photographs with their evaluations and have them compete against each other so that they have to achieve their health goal faster in a given amount of time. This will add the competitive nature that is popular in many successful games. 


\section{Conclusion}
We piloted Monster Munch (N=68) as a feasibility-focused study to investigate two major gamification mechanisms: player avatar identification and crowdsourced community board. Though avatars appear frequently as a game mechanic in games and apps with gamification, they are not often paired with crowdsourced social computing platforms and investigated in conjunction with learning. From our pilot study, we witnessed that these mechanisms were not only accepted by the users, but also they encouraged them to engage with macronutrient nutrition topics based on crowdsourced meal photographs. Though the avatar bond from the user was negatively associated with nutrition learning, the pilot highlights these mechanisms as effective tools to further discuss lightweight approaches for nutritional engagement. Future studies should investigate the right combination of gamification mechanisms that can promote nutritional engagement to improve nutritional literacy.


%%
%% The acknowledgments section is defined using the "acks" environment
%% (and NOT an unnumbered section). This ensures the proper
%% identification of the section in the article metadata, and the
%% consistent spelling of the heading.
\begin{acks}
This is partly funded through ABC University's hungry hungry monster grant.
\end{acks}


%%
%% The next two lines define the bibliography style to be used, and
%% the bibliography file.
\bibliographystyle{ACM-Reference-Format}
\bibliography{acmart}
%sample-base

%%
%% If your work has an appendix, this is the place to put it.
\appendix

% \section{Research Methods}

% \subsection{Part One}

% Lorem ipsum dolor sit amet, consectetur adipiscing elit. Morbi
% malesuada, quam in pulvinar varius, metus nunc 

% \subsection{Part Two}

% Etiam commodo feugiat nisl pulvinar pellentesque. Etiam auctor sodales
% ligula, non varius nibh pulvinar semper. Suspendisse 

% \section{Online Resources}

% Nam id fermentum dui. Suspendisse sagittis tortor a nulla mollis, in
% pulvinar ex pretium. Sed interdum orci quis metus euismod, et sagittis


\end{document}
\endinput
%%
%% End of file `sample-sigplan.tex'.
